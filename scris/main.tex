\documentclass{article}
\usepackage[utf8]{inputenc}
\usepackage{titlesec}
\usepackage{subcaption}
\usepackage{multicol}
\usepackage{hyperref}
\usepackage{amsmath}
\usepackage{mathtools}
\usepackage{listings}
\usepackage{color}
\usepackage{indentfirst}
\usepackage[normalem]{ulem}

\captionsetup{compatibility=false}
\definecolor{orange}{RGB}{255,72,0}
\definecolor{dkgreen}{rgb}{0,0.6,0}
\definecolor{gray}{rgb}{0.5,0.5,0.5}

\lstset{frame=tb,
  language=Python,
  aboveskip=3mm,
  belowskip=3mm,
  showstringspaces=false,
  columns=flexible,
  basicstyle={\small\ttfamily},
  numbers=none,
  numberstyle=\color{blue},
  keywordstyle=\color{orange},
  commentstyle=\color{dkgreen},
  stringstyle=\color{dkgreen},
  breaklines=true,
  breakatwhitespace=true,
  tabsize=3
}

\hypersetup{
pdftitle={thesis},
pdfauthor={Ciprian},
pdfkeywords={pdf, latex, tex, ps2pdf, pdflatex},
bookmarksnumbered,
pdfstartview={FitH},
urlcolor=cyan,
colorlinks=true,
linkcolor=red,
citecolor=green,
}


\font\myfont=cmr12 at 24pt
\font\myfontE=cmr12 at 18pt
\font\myfontS=cmr12 at 16pt
\title{\textbf{BABEŞ-BOLYAI UNIVERSITY CLUJ-NAPOCA \\ \vspace{1ex} 
FACULTY OF MATHEMATICS AND COMPUTER SCIENCE \\ \vspace{1ex}
SPECIALIZATION COMPUTER SCIENCE \vspace{5ex}}}
\author{\myfont Diploma thesis}
\date{\myfontE Focus detector with Convolutional neural network}

\setcounter{secnumdepth}{4}
\usepackage{natbib}
\usepackage{graphicx}
\titleformat{\paragraph}
{\normalfont\normalsize\bfseries}{\theparagraph}{1em}{}
\titlespacing*{\paragraph}
{0pt}{3.25ex plus 1ex minus .2ex}{1.5ex plus .2ex}



\begin{document}

\clearpage\maketitle
\thispagestyle{empty}

\vspace{32ex}
{\myfontE {\hspace*{3em}Supervisor:} {\hspace*{8em}Author:}}
\newline
{\myfontE Prof. Dr. DIOSAN Laura {\hspace*{3em}Cuibus Ciprian}}
\begin{center}
    \vspace{3ex}
    {\myfontE 2019}
\end{center}

\pagenumbering{roman}

\textcolor{green}{\sout{Sa adaugi in partea de jos a primei pagini si anul (2019)} si sa incerci sa formatezi prima pagina ca in exemplul de coperta de pe site-ul cs}

\newpage
\begin{center}
    \textit{\textbf{ Abstract}} 
\end{center}
\textit{ In a world, where technology had become 'a must' in the human life, the attention had shifted towards platform like Facebook, Instagram, Reddit and so on. When this distraction is combined with the driving action, some catastrophic accidents occurs. This thesis gives, a system that can be used on cars, in order to minimize the distractions, of all sorts(e.g. live streaming on Facebook), while driving. The systems learns how to detect and alert if the driver is not focused. The thesis contains two main parts, one begin the theoretical concepts and the other one explaining how the system was constructed.}

\textcolor{green}{\sout{tb si un abstract (o poveste plastica, ne-tehnica, "pt bunica") care sa contina ce problema rezolvi si utilitatea ei, cum o rezolvi, cat de bine ai reusit sa o rezolvi si structura lucrarii (similar cu finalul introducerii, dar poate mai scurt)}}

\textcolor{green}{\sout{cred ca ai putea reorganiza sectiunile astfel:\\
1. Introducere\\
2. AI in automotive (actuala 2.3)\\
3. Theory (actuala 2.1 si 2.2) \\
4. Focus detector (Actuala 2.4, separat in 2: state of art si abordarea ta)\\
5. App (si aici inainte de sectiunea 3.4 Experiments sa ai o sectiune de Testing and validation (asserts, unittesting, coverage, etc.)\\
6. Conclusions}}

\tableofcontents
\listoffigures
\listoftables

\newpage
\pagenumbering{arabic}
\section{Introduction}
    \subsection{Importance \&\ Motivation}
        It has been a real problem for the people of the twenty-first century when it comes to being safe while behind the wheel. Since the introduction and popularization of the mobile phone, the majority of the population shifted their attention to the small screen of their phones, thus putting themselves in danger. This habit has quickly made it's way into driving, making the drivers less focused to the road, thus causing more accidents and even worst death of some unfortunate people. A driver has to be focused at many things, such as road signs, pedestrians and many more, by paying more attention to the mobile phone you expose yourself to a possible driver error \cite{trafficConflict}. Also, as the time goes on the automakers keep making cars which are more powerful and faster, thus, taking into account the lack of attention caused by the mobile phones and you have a good recipe for a car accident. As Jeremy Clarkson said “Speed has never killed anyone. Suddenly becoming stationary, that's what gets you.” (goodreads). This may serve as a useful tip for drivers around the world, because not the speed is the bad factor, but the lack of attention to the road and surrounding causes accidents which can lead to dead people.    
\newline
One of the problems to deal with is the rate of distraction and how a specific distraction can turn into a driving error \cite{trafficConflict}. Assuming that a distraction turns out into a specific mistaken driving pattern, than the rate of distraction \ensuremath{\lambda} can be a variable which evaluates the number of distractions that a certain driver can undergo while driving for a fixed period of time \cite{trafficConflict}. The rate \ensuremath{\lambda} is not sufficient to describe the distraction event, so a new variable must be assumed, a distribution \ensuremath{\psi} of the profile of erroneous driving pattern \cite{trafficConflict}. In theory, if the rate \ensuremath{\lambda} and the distribution of \ensuremath{\psi} is known, a Monte Carlo experiment could be adapted to initiate randomly the immediate of distraction and the resulting mistaken driving pattern \cite{trafficConflict}. 
\newline
A fundamental speculation for this rate-of-distraction-based model could be  that the crash frequency is strongly connected to how frequent the driver is distracted, which may be supposed to occur randomly and uniformly across any driven distance \cite{trafficConflict}. In other words, at any moment(or point) of driver's trip, it has the same probability of making an error \cite{trafficConflict}. Nevertheless, some of the crash data have shown that crash rates are higher at specific locations and in specific traffic flow surroundings \cite{trafficConflict}.
\newline
A similar approach has been in the brand new line of cars from BMW, which has a cam in the dashboard that monitorize the behaviour of the driver behind the wheel.
\newpage
    \subsection{Objectives}
        The thesis aims at studying and developing a focus and drowsiness detector in order to improve safety in traffic by making the driver more focused at the road. The openCV approach consists of algorithms that detects the face and eyes which are the key for accurately detect whether the driver is distract by something or is attention is on the road.
    \subsection{Thesis structure and original contribution(s)}
        The research in this dissertation advances the theory, design and implementation of several particular models. The present work contains 17 bibliographical references and its structured in four chapter as follows:
\begin{itemize}
  \item The first chapter is a short introduction to the problem that this thesis is covering and what objectives are stated. 
  \item The second chapter is a quick overview of how artificial intelligence played an important role in developing the cars of the future.
  \item The third and fourth chapter describes the theoretical part of the thesis which consists in making a short introduction to AI, convolutional neural networks followed by a brief explanation of the influence of the artificial intelligence in automotive and a brief explanation of the technical part of the application. The technical part will parse through how the face and eyes are detected and how the drowsiness detector works.
  \item The fifth chapter covers the implementation of the application which frameworks were used, how the architecture was made and what tests were performed.
  \item The sixth chapter and the last one will be represented by the conclusion of this thesis. \textcolor{green}{\sout{ultimul capitol tb sa fie cel cu conclusions si further work (sectiunea cu bibliografia nu se aminteste in structura tezei; e oarecum implicit ca o lucrare de gen sa aiba biblio :D}}
  
\end{itemize}

\newpage
\section{Artificial intelligence in automotive}
    The first interaction between an intelligent system and the automotive industry was with the radio controlled car (Houdina Linriccan Wonder). On the rear of a 1926 Chandler were mounted antennae and it could be controlled by another car that sent out radio impulses while coming after the Houdina Wonder. The signals were sent to the "circuit-breakers" which control some small electric motors resulting in directing the car's actions \cite{AutoAI}. This try marked a basic form of what is called today autonomous vehicle. In 1939, General Motors (GM) had sponsored Normal Bel Geddes's Futurama display at the "World's Fair" illustrating an embedded-circuit powered electric car. Like the previous tries, the circuits were planted in roadway and controlled by radio \cite{AutoAI}. See Figure \ref{fig:ex_autoCar}. \par

In year of 1953 on the laboratory floor of the RCA Labs was build a miniature car, that was controlled and guided by cables that were places in a specific pattern. Leland Hancock (state traffic engineer in Nebraska) and L. N. Ress (state engineer) tested the idea of RCA Labs on an actual highway installation. The test took place near the town of Lincoln (Neb), on a 121.92 meters long portion of highway, in the year 1958. On the surface of the road were placed some detector circuits and lights on the edge of the road, that were capable to send impulses to control the car. GM was an important collaborator of the project and equipped two models with sophisticated radio receivers, audible and visual warning devices capable to mimic automatic steering, accelerating and brake control \cite{AutoAI}. \par 

During the 60's the United Kingdom's Transport and Road Research Laboratory tested a driverless car, that went at a speed of 130 km/h, on a test track, without aberrations of speed or direction. Also the car interacted with magnetic wires that were integrated in the road \cite{AutoAI}. \par

\begin{figure}[h!]
    \centering
    \includegraphics[width=1\linewidth]{Images/230px-Linrrican_Wonder.png}
    \caption[Houdina Linriccan Wonder car]{Houdina Linriccan Wonder car \protect\footnotemark}
    \label{fig:ex_autoCar}
\end{figure}

\footnotetext{image taken from: https://en.wikipedia.org/wiki/Houdina\_Radio\_Control\\\#/media/File:Linrrican\_Wonder.png}

After two decades in the 80's, Mercedes-Benz released a robotic van designed by Ernst Dickmanns and his team within the Bundeswehr University Munich, Germany. The robotic van reached a speed of 63 km/h on no traffic roads. Furthermore, different national and international projects, thanks to the progress in autonomous cars, were launched. Between 1987 and 1995, a project named Prometheus conducted by EUREKA was started with one goal in mind, to advance further in the field of autonomous cars. In this project were invested over 1 billion US dollars. Also in the same time period, DARPA (defence advanced research projects agency) of the US Department of Defence had developed the first autonomous land vehicle (AVL) that achieved road-following using computer vision, LIDAR and autonomous control to help the robot to achieve a speed of 31 km/h. These technologies were developed within Carnegie Mellon University, the Environmental Research Institute of Michigan, University of Maryland, Martin Marietta and SRI International. \cite{AutoAI}. \par

One of the most important projects of the late twenty century happened in 1995, the Carnegie Mellon University’s Navlab project had achieved an autonomous driving percentage of 98.2\%\ on a 5,000 km trip, known as "No Hands Across America" or NHOA. Although the car was not fully-automated, only the control of the steering wheel was done by using neural networks, but the throttle and brake were operated by a human \cite{AutoAI}. \par  

One year later, Alberto Broggi started the ARGO Project. The goal of the project was to travel 1900 km in six days in northen Italy, just following the the painted lane marks and with an average speed of 90 km/h. Overall, the car managed to be 94\%\ of it's journey in fully-automatic mode and the longest automatic stretch of 55 km \cite{AutoAI}. \par 

In the early 2000s the US government start implementing and funding autonomous military vehicles, such as Demo I, Demo II and Demo III. These autonomous robots are able to travel kilometers on difficult terrain, detect and avoid obstacles like rocks, trees etc, making the soldiers life easy \cite{AutoAI}. \par

The year of 2010 mark the unveiling of the first 2 seat electric car that could be operated manually or autonomously, at the Expo 2010 in Shanghai. GMs 2 seat electric car, know as Electric Networked Vehicle, was divided into three distinct vehicle types. GM's ENV quickly became an important asset in making vehicles more connected with each other and motion control algorithms. From July 20, 2010 to October 2, 2010, the VIAC or VisLab Intercontinental Autonomous Challenge took place. This challenge was a 13,000 km trip from Parma, Italy to Shanghai, China. The goal of this prject was to show that in the future goods could be transported between continents with vehicle that require slight intervention by the human. Another, important, milestone was the Audi's TTS autonomous research car, that completed, in September 2010, Pike's Peak in around 27 minutes, fully autonomous, coming about 10 minutes of delay to the human record of 17 minutes. This was done by using a similar approach to the auto pilot feature of the airplanes. Also, the project was sponsored by the EU as part of the HAVEit (Highly Automated Vehicles for Intelligent Transport) project \cite{AutoAI}. \par

The first car that was licensed on the public streets and highways of the German state of Berlin were "MadeInGermany" and "Spirit of Berlin", both developed by the AutoNOMOs Labs. Some of the technologies developed for this project were: driver assistance system, an innovative safety system. It was equipped with an accurate GPS unit and three laser in the front and back that could detect pedestrians or cars 360 degrees, traffic lights, intercity traffic and roundabouts. Those from Daimler R  \&\ D in collaboration with Karlsruhe Institute of Technology made a Mercedes S-Klasse capable to drive autonomously for a distance of 100 km from MannHein to Pforzheim, Germany. It used a lot of technology such as next generation radars and stereo cameras which assisted in its "autonomous automation" and cutting edge machine vision algorithms. Also, Toyota had developed an autonomous car focused in negating the crashes, they used ITS, Intelligent Transport Systems, technology. The systems of the car were engineered in such a way that in case of a failure the car will not crash. Although the car is not fully-autonomous, the human being able to intervenes at any time \cite{AutoAI}.  \par

BRAiVE, one of the autonomous cars that participated in VIsLab, drove, in July 12th 2013, in the center of Parma. It was the first fully-autonomous car that navigated on rural roads, crosswalks, traffic lights, pedestrian areas, roundabouts and artificial hazards \cite{AutoAI}. \par

The Induct Technology from France had made a robotically guided electric shuttle named "Navia". The robot can reach a maximum speed of 20 km/h and can fit 10 passengers. It was tested in some universities across England, Singapore and Switzerland. It was capable to generate a 3D map of the surrounds thanks to stereoscopic cameras and four LIDAR units \cite{AutoAI}. Lastly the giant Google had make his way into developing and improving the software, in order to create fully-autonomous cars capable to surpass the humans and make the roads more safer than ever. \par

The focus nowadays, in the driving domain, is the design of perception and control systems, that uses learning based approaches with large data collections and annotations in order to build models that are able to operate over edge cases of some real-world operations. Thus, making the deep learning research in focuses to address detection, estimation, prediction, labeling, generation, control and tasks planning. Some of the current technologies in developing are "body pose estimation", "semantic scene perception", "fine-grained face recognition" and "driving state prediction" \cite{Vision, MIT}. These tasks are of a great importance because it can lead to understand how to mimic the human posture and to create an awareness based on the current environment. In what is to come, it will be briefly summarized the above states technologies as follows: 

\begin{itemize}
    \item \textbf{Fine-grained Face Recognition}: This type of face recognition take a step forward, ahead of classical face recognition techniques, by focusing on understanding the behavior of a human toward recognizing the face, like facial expression or detect the eye gaze. This technique is very useful to explore, in depth, the power of predicting the driver eyeballs in a way that makes the driving more secure and the driving experience more pleasing \cite{MIT, Frust, DetectEmotional}.  
    
    \item \textbf{Body Pose Estimation}: By using this approach it can enhance the performance, efficiency and understanding of a lot of real-world usage for robots and action recognition. Some of the fortunate approaches range from the usage of "depth image",via DNN (deep neural network) or the usage CNN (convolutional neural network) and graphical models. This technique is especially useful for driving, which detects the driver posture, presented by the "skeleton data", that includes the position of the wrist, shoulder joints and elbow in order to model the behavior of the humans when driving. Thus, monitoring the vigilance of the driver through "visual analysis of eye state and head pose" \cite{MIT}.
    
    \item \textbf{Semantic Scene Perception}: Interpreting and understanding the picture from a 2D image has been a long and challenging task in computer vision, this, usually is referring to the autopsy of the semantic image. By using datasets of  large proportions (datesets of places, cityscapes etc) and mighty deep learning approaches obtaining some state-of-the-art results. Thus, the study of precise "driving scene perception" is nowadays actively considered in both universities and automotive industry \cite{MIT, EndTOEnd}.
    
    \item \textbf{Driving State Prediction}: Represents the goal of autonomous driving, the state of the vehicle is commonly considered as a "direct illustration of human decision" while driving. For this, the machine learning approach serves as the "ground truth" for a lot of distinct viewpoints, like driving behavior, steering commands, controlling the speed etc \cite{MIT, EndTOEnd}.
\end{itemize}

In the modern era of vehicles thinks like driver assistance, vehicle performance and driver experience are more and more automated using learning-based ways, thanks to more datasets being released that could be used by communities of researchers. One study from MIT, MIT-AVT, that aims to be the base of countless such datasets, which helps developing and train new neural network architectures in order to bring current and future solutions for, a lot of standard and 
mixed subtasks, that helps semi-autonomous or fully-autonomous driving be more safe and affordable \cite{MIT}.
\newpage
\section{Theoretical Part}

This chapter will present some historical thoughts about AI, followed by a short overview of machine learning. Moreover, a deep dive is made on neural network, convolutional neural network and how they work.

\textcolor{green}{\sout{arata mai bine daca intre titlul capitolului si prima sectiune ai adauga 2-3 propozitii despre ce e vorba in capitolul curent (un mic rezumat)}}

    \subsection{Introduction to AI}
        The roots of Artificial intelligence, and the concept of smart machines, can be found in the Greek mythology. Some intelligent devices have appeared, since then, in the literature, that have demonstrated to behave with a degree of intelligence. Following the World War II which marks the beginning of the so called era of "modern computers". This machines were able to create programs that perform difficult intellectual tasks. Using these programs, humans have created general tools with a wide variety of applications in everyday problems \cite{briefAIHistory}, see Figure \ref{fig:SRI_Robot} and \ref{fig:Unimate_Robot}.

\begin{figure}[h!]
    \centering
    \includegraphics[width=.3\linewidth]{Images/SRI_Shakey.jpg}
    \caption[Shakey the Robot]{Shakey the Robot was the first general-purpose mobile robot to be able to reason about its own actions. While other robots would have to be instructed on each individual step of completing a larger task, Shakey could analyze commands and break them down into basic chunks by itself.\protect\footnotemark}
    \label{fig:SRI_Robot}
\end{figure}

\footnotetext{image taken from: https://en.wikipedia.org/wiki/Shakey\_the\_robot\#/media/File:\\SRI\_Shakey\_with\_callouts.jpg}

\begin{figure}[h!]
    \centering
    \includegraphics[width=.4\linewidth]{Images/UnimateRobot.jpg}
    \caption[Unimate]{The Unimate was the first industrial robot, used at General Motors assembly line. It was invented by George Devol in the 1950s.\protect\footnotemark}
    \label{fig:Unimate_Robot}
\end{figure}

\footnotetext{image taken from: https://www.robotics.org/joseph-engelberger/unimate.cfm}

As stated above, artificial intelligence is the intelligence that is manifested by machines, in contrast to the natural intelligence displayed by humans or animals. In computer science the research of artificial intelligence is defined as the study of 'intelligent agents' \cite{IntelliAgents}. Moreover artificial intelligence can be classified into three categories of systems.: analytical, human-inspired and humanized AI. The first one, Analytical AI, generates a cognitive image of the world by learning based on past experiences to predict future choices.  Human-inspired AI is able to understand human emotions, in addition to the cognitive ones, by combining elements from both cognitive and emotional intelligence. The last type is called Humanized AI which is a machine capable of being self-conscious and self-aware in interactions with others, having a all types of competencies (i.e. cognitive, emotional and social intelligence) like his human counterpart. \par

One of the main core part of the artificial intelligence is Machine learning. Most of the science of a machine learning system is to solve problems and give good insurance for the end result \cite{MLIntro}. Learning without any kind of supervision requires an ability to identify patterns in streams of input. \par

Some of the most common use of the machine learning is in the implementation of a search engine e.g the Google search from Google. This means that when searching on Google, the engine behind receives a query and returns the relevant webpages. To achieve this goal, any search engine has to 'know' which pages to return based on their relevancy and query (see Figure \ref{fig:SearchEngine}). This knowledge can be gained from two main sources: the first one is based on the link structure of webpage, content and the frequency of what the users will follow on the suggested links in a query, and the second source is to compare the entered query with existing queries. For a more efficient search, machine learning is used increasingly to automate the process of designing a search engine \cite{MLIntro}. \par

\textcolor{green}{\sout{ai te rog grija cu parantezele: inainte de paranteza se lasa un spatiu liber, dupa paranteza nu se lasa; cateva situatii am corectat eu, dar te rog verifica tot textul}}

Other applications that take advantage of machine learning are speech recognition (being able to translate the speech in text, such as the Google service when shout 'Ok Google'), finger print recognition (for security purpose, being able to know which finger print is, widely use in securing phones), safety features on modern cars (such as adaptive cruise control, being able to adapt the speed based on the car in front,  lane keep, the ability to stay within the same lane and adaptive headlights, the capability of the lights to turn of when detecting a incoming car to not blind the other driver), making the behaviour of the avatar in computer games (i.e Formula 1) \cite{MLIntro}. \par

\textcolor{green}{\sout{ucred ca ar merita sa amintesti pe scurt si de cele 3 directii mari din ML: supervised learning, unsupervised learning si reinforcement learning}} \par
In machine learning there are three main paradigms:

\begin{itemize}
    \item The first paradigm is \textbf{supervised learning}, which translated in real world it assumes only knowing the starting and ending point of a journey, and the route will be determined progressing. So in supervised learning the training is done through pre-labelled inputs. Each training sample has a set of input values and multiple (one or more) associated labeled output values. The goal of this technique is to cut down the classification error, by correct computation of the resulting value of a specific training example, in the training phase. \cite{IntroCNN}.

    \item The second is \textbf{unsupervised learning} paradigm, which means that the input data is no more labeled. Therefore, the success is given whether the network can reduce or increase the linked cost function. However, for a network that uses an unsupervised learning technique it's important to note that in order for this paradigm to work properly, a supervised learning must be done for small portion of the given data \cite{IntroCNN}.  
    \item The last important paradigm is \textbf{reinforcement learning} which represents the learning of what to do, how to tie up situations to actions, in order to boost the numerical knowledge. Thus, unlike supervised or unsupervised learning, the output is given by trying the actions that gives the better rewards. In reinforcement learning, the trial and error and delayed reward are the most essential features \cite{Reinforcement}. 
\end{itemize}

The main difference between a machine learning and a computer program, consists of what the end result will look like. In other words, a computer program takes data as input, processes them and finally gives an output or result.Also a programmer will impose how the data is processed. On the other hand in machine learning, the input and output data is given, and the machine will find the processes with which the given input achieves the given output data. Moreover, using this process the machine can predict the unknown output, when new input data is given \cite{ANNBasic}.

The number of problems that can be solved using learning is by no means small, as stated in the lines above. In other words, the number of templates identified by the researchers which makes the deployment of machine learning quite easy is growing rapidly. In the following lines there is a list by no means complete of templates \cite{MLIntro}:

\textcolor{green}{Enumerarile de mai jos ar putea fi formulate folosind aceleasi filtre:\\
1. ce se da si ce se cere\\
2. care sunt diferentele si asemanarile intre ele\\
\\
De exemplu au as vorbi mai intai de specificarea unei probleme de ML: se dau date (in si out) si se cere identificarea unei legaturi/pattern intre in si out; apoi as zice ca la clasificare out-ul sunt label-uri (2 --- binary classification --- sau mai multe --- multiclass classification), iar la regresie out-ul este numeric; apoi la structured estimation out-ul e reprezentat de laebl-uri si inca ceva informatii suplimentare
}

\begin{itemize}
  \item \textbf{Binary Classification} probably the most often studied problem in machine learning, which has led to a large number of relevant algorithmic and theoretic improvements over the last century \cite{MLIntro}.
  \item \textbf{Multiclass Classification} which represents an extension of the binary classification. The main difference is that in the case of multiclass there are a range of different values \cite{MLIntro}.
  \item \textbf{Structured Estimation} goes a step above the simple multiclass estimation by taking into account that labels have secondary structure which can help in the estimation process \cite{MLIntro}.
  \item \textbf{Regression} presents another prototypical application. The goal of this template is to estimate a real-value variable for which a pattern is given \cite{MLIntro}.
  \item \textbf{Novelty Detection} is a bit vague template. Its used to describe the problem of resolve 'unusual' perception given a set of past measurements \cite{MLIntro}.
\end{itemize}

\begin{figure}[h!]
    \centering
    \includegraphics[width=.8\linewidth]{Images/SearchEngine.png}
    \caption{The 5 top scoring sites from the query 'machine learning' \cite{MLIntro}}
    \label{fig:SearchEngine}
\end{figure}

In order to measure and evaluate the performance of a machine learning algorithm, some performance metrics can be used. Every metrics will influence, in a different way, how the machine learning algorithm will be measured and compared. 

The first metric is called \textbf{Confusion Matrix}, being the easiest one. It is used for discovering the accuracy and faultlessness of a model. Moreover, is good for classification problems in which the output is of two or more types of classes \cite{Metrics}. Terms that correlate with the first metric (for simplicity 1 is True and 0 is False for a binary classification): 

\begin{figure}[h!]
    \centering
    \includegraphics[width=.6\linewidth]{Images/ConfusionMatrix.png}
    \caption{Confusion Matrix \cite{Metrics}}
    \label{fig:ConfusionMatrix}
\end{figure}

\begin{enumerate}
    \item \textbf{True Positive (TP):} occurs when the expected output is \textbf{1} and the value predicted is also \textbf{1} \cite{Metrics}.
    \item \textbf{True Negative (TN):} manifests when the expected output is \textbf{0} and the result of the prediction is \textbf{0} \cite{Metrics}.
    \item \textbf{False Positive (FP):} represents a mismatch between the actual result of the class and the result of the prediction, namely the expected output is \textbf{0} and the prediction gives \textbf{1} \cite{Metrics}.
    \item \textbf{False Negative (FN):} is also a mismatch of the actual result and the predicted one, namely the expected value is \textbf{1} and the predicted is \textbf{0} \cite{Metrics}.
\end{enumerate}

The second metric, which is very useful, is \textbf{Accuracy}. In classification problems represents the number of correctly made predictions over the total number of predictions \cite{Metrics}.

\begin{figure}[h!]
    \centering
    \includegraphics[width=.6\linewidth]{Images/Accuracy.png}
    \caption{Accuracy \cite{Metrics}}
    \label{fig:Accuracy}
\end{figure}

Where at the nominator are the correct prediction, whereas at the denominator are all the predictions made \cite{Metrics}.

\textbf{Precision} is the next performance metric that measures the proportion of the data that has been predicted as True (TP and FP) and the data that are actually true (TP) \cite{Metrics}. 

\begin{figure}[h!]
    \centering
    \includegraphics[width=.6\linewidth]{Images/Precision.png}
    \caption{Precision \cite{Metrics}}
    \label{fig:Precision}
\end{figure}

\textbf{Recall or Sensitivity} is another metric that measures the proportion of data that is actually True was determine by the algorithm as True. The positives (TP and FN) and the data 'diagnosed' by the model as True (TP) \cite{Metrics}.

\begin{figure}[h!]
    \centering
    \includegraphics[width=.6\linewidth]{Images/Recall.png}
    \caption{Recall or Sensitivity \cite{Metrics}}
    \label{fig:Recall}
\end{figure}

\textbf{F1 Score} represents the Harmonic Mean betwixt precision and recall. It tells how clear-cut the classifier is and how powerful it is \cite{Evaluate}. Mathematically, it can be declare as follows:

\begin{center}
    \begin{equation}
        F1 =2 * \frac{1}{\frac{1}{precision} + \frac{1}{recall}}
    \end{equation}
\end{center}

\textbf{Mean Absolute Error (MAE)} represents the mean of the difference betwixt the original and predicted values. It returns the measurement of how distant are the predictions from the expected output. \textbf{Mean Square Error (MSE)} is very similar to Mean Absolute Error with the difference that the MSE takes the mean of the square of the computed difference betwixt the original and predicted values. For computing the gradient MSE is more suited than MAE, because does not require complex computations \cite{Evaluate}.

\textcolor{green}{\sout{eu as aminti acest aspect legat de date si process inainte de clasificare problemelor de ML si as include cu o enumerare a algoritmilor care pot fi folositi in ML, precizand ca in sectiunea urmatoare se va detalia alg de ANN, iar apoi as incheia sectiunea cu o prezentare a metricilor de performanta care se folosesc in ML (MSError, Accuracy Precision, Recall, F-measure, True positive, False positive, etc. - metrici independente de alg de learning folosit in rezolvarea unei probleme}}

    \subsection{Neural network}
        \subsubsection{Background}
            The most basic unit by which the nervous system works, in terms of processing is the neuron. They communicate with each others via synapses, by sending electrical signal (axon is the longer part which sends information to the dendrites of the other neuron). Through this connections the electrical signal is transmitted  across multiple neurons. The brain of a human is known to have a lot of neurons, approximately 100 billion neurons. If it is the case to reproduce this amount using existing computers, than it will be almost impossible to imitate this level of complexity. \textcolor{green}{oare stii de proiectul acesta: http://www.humanconnectomeproject.org/ ?}
As a reference a nematode has around 302 neurons and by using today technology would be easy to replicate this level of complexity \cite{ANNBasic}.\par

As stated above neurons receive, process and give an output data. However, the output isn't given at a constant rate, like an asynchronous call to get some data from the backend, unless it reaches a certain threshold. In other words, the function that does the above after a certain threshold is called an activation function \cite{ANNBasic}. 

\begin{figure}[h!]
    \centering
    \includegraphics[width=1\linewidth]{Images/exampleofthreshhold.png}
    \caption{A step and sigmoid function \cite{ANNBasic}}
    \label{fig:ex_threshold}
\end{figure}

As shown in Figure \ref{fig:ex_threshold}, the artificial neurons are very similar with the biological ones, in the way they respond to the stimuli. Moreover, in reality, artificial neural networks (ANNs) use , besides the action functions, some other functions called logistic functions. At the moment the most used function in the learning process is the sigmoid function, because is not so difficult to calculate in comparison with other functions. Also, in the course of chancing the weight values, the hole layer needs an activation function which can be differentiated \cite{ANNBasic}.The sigmoid function can be expressed as follows: 

\begin{center}
    \begin{equation}
        \ensuremath{\sigma}(x) =\frac{1}{1 + e^{-x}}
    \end{equation}
\end{center}

Both biological neurons and those from an artificial neural network can receive multiple inputs \cite{clinical}. The difference is the neurons in an artificial network need to add them in order to exceed the threshold and the processing is done using an activation function (i.e sigmoid function). Whereas a biological one simply process the given information \cite{NNandML, ANNbriefOverview}. The result processed by the activation function, becomes the output value for the next layer of neurons. As shown in the Figure \ref{fig:ex_input}, if the value resulted by adding the inputs A,B and C is greater than the threshold, this neuron will generate an output value  \cite{ANNBasic, MakeyourOwn}, as seen in the Figure \ref{fig:ex_input}.

\begin{figure}[h!]
    \centering
    \includegraphics[width=1.4\linewidth]{Images/exampleOfInputOutput.png}
    \caption{Input and output of data from a neuron \cite{ANNBasic}}
    \label{fig:ex_input}
\end{figure}

Typically, in a biological neural network, neurons are located over several layers, and one neuron can have multiple connection with other neurons. The same architecture is applied in an artificial neural network, but with the following differences, the first layer, called the input layer, composed by input neurons which sends information through synapses to the second layer, called hidden layer (in this layer the programmer is unable to interpret how the result is obtained), and finally the third layer (output layer), which receives data from the second layer via more synapses \cite{MakeyourOwn, RealTime, ANNbriefOverview, NNandML}. \par

The learning process for an artificial neural network is done by updating some variables, such as the strength between the nodes, called weights, in order to obtain better matched output data to the target ones. \textcolor{green}{\sout{eu as evita sa folosesc termenul de accuracy pt ca nu e inca explicat/introdus si e oarecum specific problemelor de clasificare (ori ANN-urile rezolva si regressi); as zice ceva de genul: in order to obtained better matched/fitted output data to the target ones}} See Figure \ref{fig:ex_weight}, also a low weight value weakens the knowledge, whereas a high weight enhance it \cite{MakeyourOwn, NNandML, ANNbriefOverview, ClassifNN}.
\textcolor{green}{\sout{cred ca ar fi mai ok ca in loc de signal (termen oarecum specific NN reale, nu artificiale) sa zici information/knowledge/data}}

\begin{figure}[h!]
    \centering
    \includegraphics[width=0.9\linewidth]{Images/exampleOfWeights.png}
    \caption{Connection with weights between neurons of each layer \cite{ANNBasic}}
    \label{fig:ex_weight}
\end{figure}

As the Figure \ref{fig:ex_weight} shows, the weights (W1,2), (W1,3), (W2,3), (W2,2) and (W3,2) are accentuated by the strength of the signal thanks to a high weight value, also if the weight is equal to 0 than the signal is not transmitted. Updating the weights is done by determining the error between the predicted and correct output \cite{ANNBasic}. Thus, the error is splitted by the report of the weights on the links, then the errors obtained are backpropagated and reassembled \textcolor{red}{\sout{the error is obtained by dividing the error to the ratio of the weights on the links, than backpropagate and reassemble the divided errors}} \textcolor{green}{\sout{ceva nu e ok error is obtained by dividing the error?}}, see Figure \ref{fig:ex_backprog}. Moreover, in the case of a hierarchical structure, computing all the weights mathematically is extremely difficult and computational demanding. Thus, the use of the gradient descent method is very helpful \cite{ANNBasic, OnlineApprox, NNTricks}. The gradient descent method as a technique is used to find the minimum point by utilizing a cost function, which is way to determine how well a machine learning model has performed given the different values of each parameters. So its less important the starting value of the weight \textit{W}, because it will gradually alter it in order to reach the local minimum. \textcolor{green}{\sout{cine e costul? pana acum nu ai amintit de el si cum il poti determina. ai vb doar de erorare}} Thus, without expensive mathematical computations, the gradient minimizes the error between the predicted and output value \cite{ANNBasic}.

\textcolor{green}{ar merita aici si o mica discutie despre loss si regularisation - ca tradeoff intre eroare buna si complexitatea modelului invatat de un ANN}

So this is the process with which an artificial neural network learns.

\textcolor{green}{poti enumera si de alti algoritmi de identificare a weight-urilor optime (Evol alg, Markov random fields, other optimization methods)}

\begin{figure}[h!]
    \centering
    \includegraphics[width=1\linewidth]{Images/exampleOfBackPropagation.png}
    \caption{Backpropagation of the error in order to update the weights \cite{ANNBasic}}
    \label{fig:ex_backprog}
\end{figure}

\textcolor{green}{\sout{asa cum am zis deja, eu as vb despre supervised si unsupervised inainte de povestea cu ANN pt ca poti folosi si alti alg, nu doar ANN, pt a rezolva o problema de supervis}
}
        \subsubsection{Convolutional neural network}            Convolutional neural networks (CNNs) are very similar to a traditional artificial neural network, by retaining the same logic of the neuron of self-optimise through learning. Also, as traditional ANNs the neurons still receive an input and perform an operation (e.g. from a raw input image to a final output) \textcolor{green}{\sout{nu e musai sa procesezi imagini; CNN-urile pot procesa si alt fel de date!}}, the entire convolutional network will still consider a single discreet score function (weights). Moreover, all the regular stuff developed in an ANN still applies \cite{IntroCNN, 3dconvolutional}. \par

The main difference between these two types of neural network is in their layers. As the name suggests, a convolutional neural network, contains layers of convolution which are good at dealing with complex computations. Unlike CNNs, a classic ANN does not have such layer, thus making the calculations more complicated. Accordingly, a CNN is just an ANN with a special architecture which is better, as an example, at dealing with images.
\textcolor{red}{\sout{The main difference between these two types of neural network is that convolutional neural networks are generally used in recognition of patterns within images}}. \textcolor{green}{\sout{si o ann simpla poate fi aplciata pe imagini, dar lipsa convolutiilor, duce la rezultate slabe. Eu as zice ca diferenta intre ANN clasic si CNN e tipul de layere, deci CNN-ul e un ANN cu o arhitectura speciala; in plus CNN-urile se pot aplcia si pe text (cu convolutii pe text, adica filtre 1D) - vezi http://www.jmlr.org/papers/volume12/collobert11a/collobert11a.pdf, dar si pe grafe (cu convolutii specifice grafelor/arborilor). Iar in cazul procesarii imaginilor cu ANN, procesarea cu o ANN clasica ar tine cont doar de intensitatile pixelilor, pe cand un CNN tine cont si de pozitionarea lor (cine cu cine e vecin, nu doar ce valori au pixelii).}} Therefore, allows the encoding of image specific physiognomy within the architecture, thus making this king of networks better for image-related tasks \cite{IntroCNN, 3dconvolutional}. \par

One of the notable weakness of a traditional ANN is dealing with the computational complexity needed to calculate the data from an image. However, if the dimension of the photos is not large, take as an example the MMIST dataset which contains images of 28 $\times$ 28, all in black and white (this is translated as 28 $\times$ 28 $\times$ 1 which means the first hidden layer contains 784 weights), than it will be manageable by most ANNs. Thus, when an artificial or convolutional neural network becomes too complex, than two important problems pop up. The first one being the fact that having unlimited computational power is almost impossible and the second problem is dealing with overfitting. \textbf{Overfitting} appears when a neural network is not capable to learn thanks multiple reasons (e.g. fits the training data too well and unable to react as good at new data) \cite{IntroCNN}.See Figure \ref{fig:ex_overfit}. \par

\begin{figure}[h!]
    \centering
    \includegraphics[width=0.39\linewidth]{Images/300px-Overfitting.png}
    \caption[Example of overfitting]{Example of overfitting, represented by the green line, whereas the black line represents a regularized model. Even though the green line follows better the training data, is too dependent on this data, thus making prediction difficult on new data \protect\footnotemark}
    \label{fig:ex_overfit}
\end{figure} \par

\footnotetext{image taken from: https://en.wikipedia.org/wiki/Overfitting\#/media/File:Overfitting.svg}

On the other hand, \textbf{underfitting} occurs when the model or algorithm does not fit the data well enough in the training phase. Thus, both overfitting and underfitting lead to a \textbf{poor prediction} on a new set of data.
        
        \subsubsection{CNN Architecture}
        As stated in the lines above a convolutional neural network focuses primarily on receiving as input images. Thus, the architecture has to be set up in a way to reflect the necessity of dealing with this type of data (dataset of images). Therefore, one of the main difference between the neurons from an ANN and the neurons from a CNN is that the neurons are organised into three dimensions (height, width and depth representing the spatial dimension). Taking as an example, the input will have the following shape 64 $\times$ 64 $\times$ 3 (height,width and depth), meaning that the final result would have the dimension 1 $\times$ 1 $\times$ \textit{n} (n being the number of possible classes). Whereas in a traditional ANN the input would contain only one dimension, the weight \cite{IntroCNN, Largescale}. \par

In what is has to come, it will shown what are the layers from a CNN and what are purpose. Therefore, a convolutional neural network is composed of three types of layers: convolutional layer, pooling layer and fully-connected layer \cite{IntroCNN}. A simplified architecture for classifying images from MMIST dataset \cite{MMIST} is shown in the Figure \ref{fig:ex_mmistarchitecture}.

\begin{figure}[h!]
    \centering
    \includegraphics[width=1.1\linewidth]{Images/MMISTarchitecture.png}
    \caption{A simple convolutional NN with five layers \cite{IntroCNN}}
    \label{fig:ex_mmistarchitecture}
\end{figure}

Taken the example of the CNN shown in the Figure \ref{fig:ex_mmistarchitecture}, the basic functionality can be separated into four key zones. \par

\begin{enumerate}
    \item the first layer is the \textbf{input layer} which holds the pixel values of an image \cite{IntroCNN}.
    \item the second layer is the \textbf{convolution layer} which finds the output value of the neurons which were connected to the local regions of the input by computing the scalar product betwixt their weights and the zone connected to the input volume \cite{IntroCNN}.
    \item The purpose of the \textbf{pooling layer} is to perform a downsampling of the spatial dimension of the input, thus reducing the number of parameters \cite{IntroCNN}.
    \item Last but not least, there are the \textbf{fully-connected layers}. Their duty is the same as in standard ANNs and pursuit to produce, from the activations, classes of scores which are used for classification \cite{IntroCNN}.
\end{enumerate}

By using this transformation method, CNNs can transform from input layer, using convolutional and downsampling, to class scores that are used in classification and regression \cite{IntroCNN}. The following three sections will show a deeper explanation of the main layers of a convolutional network.
    
        \paragraph{Convolutional layer}
        This type of layer is the most important one in a convolutional neural network architecture and it's parameters focuses around a learnable \textbf{kernel}. Usually the size of a kernel, in spatial dimensionality is small, but they are spreading in the entire depth of the input. Moreover, when the input information meets a convolutional layer, it transforms the data into a 2D activation map by using filters over the spatial dimensionality of the data. Thus, every kernel has an activation map, that will be stacked in the entire depth dimension in order to form the full output size from a convolutional layer. In the end,\sout{s} a scalar product will be computed for every value in a kernel. \cite{IntroCNN, EvalPooling}.\par

\begin{figure}[h!]
    \centering
    \includegraphics[width=1\linewidth]{Images/exampleOfConvolution.png}
    \caption{Representation of a convolutional layer, where the midpoint element of the kernel is located over the input vector, of which is than computed end reinstaled with a weighted bulk of itself and every adjacent pixels \cite{IntroCNN}.}
    \label{fig:ex_conv}
\end{figure}

As shown in the Figure \ref{fig:ex_conv}, the network will learn kernels that 'heats' at a specific aspect for a given spatial point of the input. In other words, these are known as \textbf{activations} \cite{IntroCNN}. \par

Every neuron in a convolutional layer is connected to a modest area of the input volume  which is referred as the \textbf{receptive field size}, in order to reduce the complexity of the model. Moreover, the magnitude of the connectedness over the depth is roughly always equivalent to the input depth \cite{IntroCNN, EvalPooling}. \par 

In a convolutional layer there are three hyperparameters (\textbf{depth, stride} and \textbf{zero-padding}) which are used to optimise and reduce the complexity of the model \cite{IntroCNN}, they are:

\begin{enumerate}
    \item The first hyperparameter is  \textbf{depth}. This parameter can influence quite heavily the output of a convolutional layer. By reducing the depth will automatically minimize the overall number of neurons in the network, but sacrificing the capabilities of the model to recognize patterns \cite{IntroCNN}.
    \item The second hyperparameter, also used in pooling layers, is \textbf{stride}. The programmer can define the stride in which to set the depth over the dimensionality of the input data for placing the sensitive field. By setting a small value for the stride (e.g. 1) will massively overlap the sensitive field, generating very large activation. In contrast, when a large value is assigned to the stride will result in a reduced amount of overlapping, producing an output of decreased spatial dimensions \cite{IntroCNN}.
    \item Lastly there is the \textbf{zero-padding} hyperparameter and it's a straightforward process of padding the input borders. Thus, being an effective approach to give more control as to the size of the output numbers \cite{IntroCNN}.
\end{enumerate}

Using these techniques will result in altering the spatial dimensionalty of the output of convolutional layers. There is a formula to compute spatial dimensionalty \cite{IntroCNN}:

\begin{center}
    \begin{equation} 
        \frac{(V - R) + 2Z}{S + 1}
    \end{equation}
\end{center}

In the above formula, the variable \textbf{V} is the input volume size (\textit{$height\times width \times depth$}), \textbf{R} is the receptive/sensitive field size, \textbf{Z} represents the chunk of zero padding set and lastly \textbf{S} is the stride. If the result obtained is not a perfect integer without decimals, than the stride is not correctly set. Therefore the neurons will not be able to fit efficiently over the input \cite{IntroCNN, EvalPooling}. \par

\textbf{Parameter sharing} takes into account that if one region feature is convenient to calculate a dimensional region set, then it's feasible to work well in other regions \cite{IntroCNN}.   
        \paragraph{Pooling layer}
            The aim of this type of layer is to reduce the resolution of the map features, where every such pooled feature coincides with the one from the previous layer. Thus, the input is combine in small $n \times n$ patch of units and the variable $n$ can have an arbitrary size \cite{IntroCNN,EvalPooling}. \par

There are two important and frequently used pooling operations: 
\begin{enumerate}
    \item \textbf{Subsampling function:} will take the sum of the inputs, multiplies it a scalar \ensuremath{\beta} after will add a bias \textit{b}, than the result will be passed over the non-linearity \cite{EvalPooling}, $a_j$ denotes the output and $a_i$ is the input of a pooling layer and N $\times$ N is the patch of the input image. It's function is the following:
        \begin{center}
            \begin{equation}
                a_j= tanh(\ensuremath{\beta} \sum_{N \times N} a_{i}^{\textit{$n\times n$}} + b)
            \end{equation}
        \end{center}
        
    \textcolor{green}{\sout{explain who are $a_i$ and $a_j$ and $N$ (big N)} \sout{and function $u$ (maybe an example)}}
    
    \item \textbf{Max pooling function:} will apply a so called window function \textit{u(x,y)}, which takes two arguments, on a input patch, after that calculates the maximum value in the neighborhood \cite{EvalPooling}, $a_j$ denotes the output and $a_i$ is the input of a pooling layer. It's function is the following.
        \begin{center}
            \begin{equation} \label{eq:max}
               a_j = \max(a_{i}^{\textit{$n\times n$}} u(n,n))
            \end{equation}
        \end{center}
\end{enumerate}
\par
In the end, regardless of what function is used, the result will be a feature map with a lower resolution. \par

The pooling layer in it's nature it's very destructive. Therefore, the stride and filters are usually set to \textit{$2\times2$}, allowing the layer to extend over the hole spatial size of the input. Moreover, there are two more types of pooling. The first one is  \textbf{overlapping pooling}, which uses a stride of 2 and a kernel which is set to 3. However, setting the kernel size to 3 will decrease the performance. The other type is the \textbf{general pooling}, being comprised of pooling neurons capable of performing a lot of common operations(i.e. L1/L2-normalization) \cite{IntroCNN}. 
        \paragraph{Fully-connected layer}
            This type of layer contains neurons that are connected directly with other neurons from two adjacent layers and they are not connected to layers within them, similar to a normal ANN \cite{IntroCNN}. 
        \paragraph{Backpropagation}
            In a convolutional neural network and also in a traditional ANN all layers will be trained with a backpropagation algorithm. The standard procedure is used in error propagation and weight variation in the following layers: fully connected, convolutional and subsampling layers. In the case of max pooling layers, the error is only propagated to a specific position using the Max pooling function, see Equation \ref{eq:max}. However, using overlapping pooling windows, it means storing more than one error signal in a single unit \cite{EvalPooling}.
    \subsection{Drowsiness detector}
\newpage
\section{Focus detector}
    In the following it will be explained in detail the theoretical part of the Focus detector, mainly how the faces and eyes are detected and passed to the CNN (convolutional neural network), which will than learn from those photos to create a good model for detecting the attention of the driver. The above sections serves as a foundation of how a neural network works and how it learns, and a brief history of the artificial intelligence in automotive and in the overall activity of the humans. As a short recap of the sections above, the artificial intelligence played a big role in making the cars more safer. In this regard, thanks to the AI (artificial intelligence) systems that could imitate the human behaviour appeared, sensors for keeping the right lane, drowsiness and much more. Moreover, the best neural network for dealing with images and their demands are, by far, the convolutional neural network. They use the following layers: convolution (responsible for learning), max-pooling (they make a downsampling for going deeper into the image) and fully-connected (similar to the layer from a traditional ANN, they serve as a consolidation of the above layers and they give the result). So, as stated in the begging of this section the main focus is to explain in detail how the face and eyes are detected by using a common technique known as haar cascades, implemented in the OpenCv framework, by Intel. Therefore, a short history and importance of this technique will be covered in this thesis also. \par

Recently, more attention was given to the Haar Cascade Classifiers (HCC) thanks to their reliability and fastness. This technique was introduced by the Viola and Jones in the early 2000s in order to detect effectively faces and landmarks of faces \cite{Viola}. This approach is based on the idea of having cascades of elementary classifiers, managed to design proper and not so computational expensive detection systems. Lienhart et al. \cite{Lienhart} in their paper explains how they improved the original haar cascade classifier by introduction new features into the pool ("rotated Haar-like" \cite{Viola}). Also, he tested how some fragile classifiers influence the achievement of the cascades \cite{Viola}. \par

Another test conducted by Meynet et al. \cite{Meynet} which combined a set of parallel weak classifiers with a relatively easy HCC. The haar cascade classifier was used to easily classify non-faces images \cite{Viola}. \par

When is the case to detect eyes, they are usually find on a localized face region. This technique is a bit trickier because detectors has to segregate betwixt eyes and some alternative facial looks \cite{Viola}. 

A study in eye detection was by the Wang et al. \cite{Wang} where they applied filters to take care of the variations in illumination. They used Support-Vector Machine and variance filters to verify the region and extract with a very good precision the location of the eyes \cite{Viola}. Some problem arrives when trying to detect multiple features from a region of a face, like eyes, nose and mouth. The authors Wilson and Fernandez \cite{Fernandez} proposed a regionalized search. This suggestion implies the knowledge of the face structure i.e. when looking for the left eye it should be located in the upper-left, similarly for the right eye in the upper-right, for the nose shoul be in the center and the mouth in the lower side of the face. This method, known in literature as Viola-Johnes algorithm, reduces by a long margin the false positive ratio \cite{Viola}. These are the main features of the \textbf{Viola-Jones} algorithm :
\begin{enumerate}
    \item Integral image: in order to detect objects as quickly as possible, using calculations of Haar features, resulting in an integral image by using few operations per pixel. After the computations are done, any other Haar feature will be computed in a constant time \cite{OpenViola}.
    \item Another important feature is having the so called "adaboost Learning" algorithm, capable of identifying and extracting only the critical features to be able to make fast and accurate classifications, discarding, at the same time, the not so important features \cite{OpenViola}.
    \item  Lastly, cascade classifiers which focuses on objects, like a human face and ignoring the background, as seen in the Figure \ref{fig:HaarCascadeClassifier}. This technique is interested more in regions of interest (roi) that may contain any object, otherwise rejecting them. A very good approach for detection in real time \cite{OpenViola}.
\end{enumerate}

\begin{figure}[h!]
    \centering
    \includegraphics[width=.8\linewidth]{Images/HaarCascadesClassifier.png}
    \caption{Some haar cascade classifiers  \cite{OpenViola}}
    \label{fig:HaarCascadeClassifier}
\end{figure}

\subsubsection{Face and eye detection}

\textbf{Haar-like features} \newline

As stated in the \cite{Lienhart}, the equation that is able to compute the Haar-like feature is show below: \cite{Haar}

\begin{center}
    \begin{equation}
        \textit{feature} = \sum_{i \in \{1...N\}} \omega_{i} \cdot RecSum(x,y,w,h,\phi)
    \end{equation}
\end{center}

\noindent
Where the function \textit{RecSum(x,y,w,h,$\phi$)} \textcolor{green}{\sout{in Eq. \ref{equation} fc RecSum are doar 4 param, aici are 5 (in ec nu apare di unghiul de rotatie)}} represents the accumulation of concentration values at any specific upright or twisted rectangle in the detection area. The parameters of the function hold the coordinates (x,y), dimensions (w,h) and $\phi$ is the rotation of the rectangle \cite{Haar}, as show in the Figure \ref{fig:DetectionWindow}.

As stated in \cite{Haar} in order to reduce the probability of having unlimited features, some restrictions should be applied as follows:
\begin{itemize}
    \item is allowed to sum pixels over at most two rectangles.
    \item the use of weights is to satisfy the difference of the area of two rectangles and to not have the same sign (i.e. if the first area is $\omega_{1} \cdot$ Area($r_{1}$)$>$0 than $\omega_{0} \cdot$ Area($r_{0}$)= -$\omega_{1} \cdot$ Area($r_{1}$)).
\end{itemize}

\begin{figure}[h!]
    \centering
    \includegraphics[width=.8\linewidth]{Images/DetectionWindow.png}
    \caption{Samples of detection windows  \cite{Haar}}
    \label{fig:DetectionWindow}
\end{figure}

Thus, taking into account the above restriction, the number of haar-features is 14 (show in the Figure \ref{fig:PrototypeHaarFeature}). This features can be scaled in any bidirectional direction and situated in any region of the detection window. Thus, the result of these features are calculated as the fraction between the sums of pixels from both black and white rectangles and than extended to satisfy the difference of the areas \cite{Haar}.

\begin{figure}[h!]
    \centering
    \includegraphics[width=.8\linewidth]{Images/PrototypeHaarFeature.png}
    \caption{All the prototypes of Haar features \cite{Haar}}
    \label{fig:PrototypeHaarFeature}
\end{figure}

For evaluating feature fast and accurate, there were introduced two innovative representations for an image: 

\begin{itemize}
    \item First representation is "Summed Area Table" (SAT(x,y)) \cite{Viola} used to compute the features using the values from the "upright rectangles" \cite{Haar}. Thus, every entry in the table is obtained as the accumulation of the strengths of the pixel over a upright rectangle stretching from the beginning (0,0) to a specified coordinate (x,y), being "filled" thanks to the following formula \cite{Haar}:
        \begin{center}
            \begin{equation} \label{eq:SAT}
                \textit{SAT(x,y)} = \sum_{x' \leq x,y' \leq y} I(x',y')
            \end{equation}
        \end{center}
    Once the table is complete, the Summed Area of that table permits the calculation of the pixel sum through any upright rectangle in just four quick visits of the equation \ref{eq:SAT} \cite{Haar}: 
        \begin{center}
            \begin{equation}
                \begin{multlined}
                    \textit{RecSum(x,y,w,h,0)} = SAT(x-1,y-1)+ \\ +SAT(x+w-1,y+h-1)- \\ -SAT(x+w-1,y-1)-SAT(x-1,y+h-1)
                \end{multlined}
            \end{equation}
        \end{center}
        
    \item Lastly there is the "Rotated Summed Area Table" \cite{Lienhart} used to compute rotated features. Entries are filled corresponding to the following formula \cite{Haar}: 
        \begin{center}
            \begin{equation}
                \textit{RSAT(x,y)} = \sum_{|x-x'| \leq y-y',y' \leq y} I(x',y')
            \end{equation}
        \end{center}
    The following formula allows to compute the pixel sum of all rotated rectangle \cite{Haar}:
        \begin{center}
            \begin{equation}
                \begin{multlined}
                    \textit{RecSum(x,y,w,h,45)} = RSAT(x-h+w,y+w+h-1)+\\+RSAT(x,y-1)-RSAT(h-x,y+h-1)-\\-RSAT(x+w-1,y+w-1)
                \end{multlined}
            \end{equation}
        \end{center}
\end{itemize}

\textbf{Classifier Cascade} \newline

The ideal classifier would be a strong one, but there are many robust classifiers, thus making the image processing take much longer than expected. To reduce the time taken to detect, a "cascade" of classifiers will be used. Constructing smaller and capable classifiers situated on the sub-windows in the given image is a good idea to improve the efficiency of the detection. In this case the strong classifier will range from the best to the worst classifier, whereas the best classifier that will have the finest feature will be capable to deny rotation, noise and negative sub-windows. More informations (rotation, noise and negative sub-windows) are ignored by the following layers called "stages" while they gather new computations \cite{OpenViola}. \par

At every stage, the Haar-like features are computed for the sub-window, after which the results are compared to a threshold, taken from the strong classifier, to check if its a face or not. When the features fulfilled the threshold condition, then the sub-window will go to the next stage from the cascade to perform the same task. If the features fails to reach the threshold in all the stages will result in rejecting the sub-window \cite{OpenViola}. \par

After the face is detected and region of interest is computed using the classifiers from OpenCV (see Figure \ref{fig:FaceEyeDetection}), the eyes are detected in the following way. Firstly the distances of the eyes are computed using the following equations \cite{OpenViola}:

\begin{equation} \label{eq:Hface}
    Hface =\text{1.8 d eye}
\end{equation}

\begin{equation} \label{eq:Heye}
    Heye =\text{0.2 h face}
\end{equation}

\begin{equation} \label{eq:Weye}
    Weye =\text{0.225h face}
\end{equation}

\begin{figure}[h!]
    \centering
    \includegraphics[width=1.1\linewidth]{Images/FaceEyeDetection.png}
    \caption{Example of how the detection works \cite{OpenViola}}
    \label{fig:FaceEyeDetection}
\end{figure}

As stated in \cite{OpenViola} all detected faces have very similar length and breadth of the eyes. Therefore, an average of the width (short wavg) and distance between eyes (short davg) can be computed using the following equation:

\begin{center}
    \begin{equation} \label{eq:Average}
        \textit{Average} = \frac{\sum_{i=1}^n xi}{n} 
    \end{equation}
\end{center}

In the case of detecting both eyes from an image the following inequality must be true \cite{OpenViola}: 
\begin{center}
    \begin{equation} \label{eq:tru}
        davg/2 < \text{distance between eyes} < 2*davg 
    \end{equation}
\end{center}

The size of the eyes may differ, therefore for every eye the equations \ref{eq:Hface} \ref{eq:Heye} and \ref{eq:Weye} in order to obtain the with of the eyes and distance between the eyes. Also, if the inequality \ref{eq:tru} is not fulfilled than the region is not taken into account. This a security measure that eliminates maliciously detected eyes \cite{OpenViola}. \par

Lastly, after the eyes were detected, all the images are transformed into arrays of pixels and passed to the neural network, which will classify the output into two section "focused" and "not\_focused". After which, the model will try to predict if the driver is focused or not, based on the prediction made. More details on implementation are discussed in the practical chapter under the implementation section.

A short description of three other algorithms or approaches to face detection or recognition will be made. The first algorithm is called \textbf{Eigenface based algorithm}, which is one of the most used approaches for face detection, thanks to it simplicity and very powerful for small datasets. The \textit{Eigenfaces} represents the characteristic feature, like eyes, nose and mouth from a face. These Eigenfaces are used for classifying the existence of a face by computing the relative distance between them \cite{Review}. \par

Another approach which is widely used is \textbf{Fisherface based algorithm} which handles mechanisms for extractions of features in face images. It tries to discover the direction of the projection in which, the images that belongs to distinct classes are detached maximally. It comes as a refinement of the Eigenface algorithm, performing better at dealing with lighting variations \cite{Review}.

The last algorithm presented in this thesis, for face detection, would be \textbf{Active Shape Model}. Its a statistical model representing the shape of some objects as restricted by point distribution model, hence the shape of the object will be decreased to, only a set of points. This approach is used to analyze images, medical images and many more \cite{Survey}. The Active Shape Model is grouped into three categories:

\begin{enumerate}
    \item \textbf{Snakes:} this technique uses active silhouette or snakes in order to locate the head boundary. Also, using these contours, features boundaries can be constructed \cite{Survey}. 
    \item \textbf{Deformable Templates:} are used to resolve the problem of detecting the face under bad lighting or contrast from an image. There are some predefined
    templates, in this method, that are used to direct the detection process, being very malleable and capable of changing the size in order to match the data \cite{Survey}.
    \item \textbf{Point Distribution Models (PDM):} represents solid and parameterized depictions of the silhouettes based on some statistics. The shape of a PDM is divided into an array of unique labeled points, where the fluctuations of them can be parameterized over a dataset which contains objects with distinct sizes and postures \cite{Survey}.
\end{enumerate}

\textcolor{green}{\sout{cred ca ar merita sa treci putin in revista si alti algoritmi de face detection (pe langa Viola-Jones). Poti vedea aici cateva recenzii:}}

$http://article.nadiapub.com/IJCA/vol8_no5/7.pdf$

$https://arxiv.org/pdf/1804.06655.pdf$
\newpage
\section{App development}
    \subsection{Problem statement + requirements}
        \textcolor{green}{\sout{teu cred ca la Requirements ar tb sa imparti in 2 task-urile: unele evidente (functionalitatile aplciatiei: monitorizare si avertizare driver) si altele ascunse (componentele si modelele inteligente (bazate pe ANN-uri)}}

\begin{itemize}
    \item \textbf{Problem statement:} Create an intelligent system that is capable to detect and warn the driver whether him/her is paying attention to the road or not and a drowsiness system that tells if the driver is about to fall asleep.
    \item \textbf{Requirements:} 
        \begin{itemize}
            \item \textbf{Functional:} \begin{enumerate}
                \item create and implement a drowsiness detector to alert the driver to not fall asleep.
                \item Monitorize at any time and check if the driver is paying attention to the road.
                \item Make sure that if the vehicle is stationary not to signal anything.
                \item If the face is not detected make sure to alert the driver.
                \item If the face is detected predict, to check whether the driver is focused, alert if is not paying attention.
            \end{enumerate}
            \item \textbf{Non-functional:} \begin{enumerate}
                \item Create and implement a neural network in order to detect if the driver is focused.
                \item Find a suitable dataset with photos of human faces, that contains a lot of images in order to have diversity and distribute them in labels.
                \item Detect the face and eyes and train the network on those images.
                \item Create and implement an user friendly UI and architecture.
            \end{enumerate}
        \end{itemize}
\end{itemize}
    \subsection{Problem design and analysis}
        \subsubsection{Architecture}
           The architecture of the focus detector consists of three files as follows:

\textcolor{green}{\sout{ar prinde bine niste diagrame aici (use cases, conceptual model, class diagram, sequence diagram)}}

\begin{itemize}
    \item \textbf{cnn.py} contains the architecture of the conlutional neural network and the image processing required for detecting the face and eye, in order to training the network (a more in depth look will be in the next section).
    \item \textbf{Utils.py} consists of functions that compute, plays or returns informations to the main file (also, a more elaborate explanation will be done in the next section). Contains the following functions: 
        \begin{enumerate}
            \item \textbf{sound\textunderscore alarm} takes a parameter, the path to an audio file and plays it.
            
            \item \textbf{eye\textunderscore aspect\textunderscore ratio} also takes one parameter that is used to compute eye aspect ratio.
            
            \item \textbf{get\textunderscore face\textunderscore utils} requires zero parameters and returns the landmarks of a face.
            
            \item \textbf{get\textunderscore haar\textunderscore cascades} returns the detection of the face and eye, also it require zero parameters.
        \end{enumerate}
    \item \textbf{main.py} contains all the computing and predictions that helps the driver to stay focused at the road. Also it has all the thresholds for any activity such as focus, drowsiness or focus prediction, all the face\&\ eye detection features in order to make the predictions (more explanations and examples are shown in the following section).
\end{itemize}

\textbf{Diagrams}

\begin{figure}[h!]
    \centering
    \includegraphics[width=0.9\linewidth]{Images/UseCaseDiagram.png}
    \caption{The use case diagram of the system}
    \label{fig:UseCaseDiagram}
\end{figure}

\begin{figure}[h!]
    \centering
    \includegraphics[width=1\linewidth]{Images/FocusDetectorSequenceDiagram.png}
    \caption{Focus detector sequence diagram}
    \label{fig:FocusDetectorSequenceDiagram}
\end{figure}

\begin{figure}[h!]
    \centering
    \includegraphics[width=1\linewidth]{Images/DrowsinessDectectionSequenceDiagram.png}
    \caption{Drowsiness detection sequence diagram}
    \label{fig:DrowsinessDectectionSequenceDiagram}
\end{figure}

\newpage
        \subsubsection{Frameworks and setup}
           A series of frameworks and libraries were used in order to develop the application. For the convolutional neural network keras under tensorflow were used, on image processing openCV was the best option and for regular computing and storage a couple of libraries were used, such as dlib, scipy, imutils and numpy. In the following a short description and explanation of why it were used are presented below: 
\begin{itemize}
    \item \textbf{Tensorflow:} the most used library for research and creating neural networks, also contains a lot of math functions. It was created by Google and it's a free and open source software, besides creating neural networks it's good for dataflow and  differentiable programming \cite{Tensorflow}.
    \item \textbf{Keras:} it's a very versatile open source framework, written in Python, capable of creating complex neural networks. It run over a lot of good deep learning frameworks such as Tensorflow, Microsoft Cognitive Toolkitm Theano or PlaidML. The main advantage of Keras is being super user-friendly and fast at the same time. It was developed to improve their research on the ONEIROS project (Open-ended Neuro-Electronic Intelligent Robot Operating System). It's name comes from Greese and it means \textit{horn}  \cite{KerasDoc, KerasBack, KerasWhy}.
    \item \textbf{OpenCV:} or open computer vision is an open source framework, for computing real-time image processing, object recognition and more, developed by Intel \cite{OpenCV}. The main language in which was written is C++, but it has interfaces in other languages like Python, Java and so on.
    \item \textbf{dlib:} a modern and fast library for improving and developing machine learning algorithms. Great for computing the movement of some facial landmarks \cite{Dlib}.
    \item \textbf{SciPy:} it comes as an extension over Numpy containing more mathematical algorithms and classes ideal for data visualization. Very useful for interpreting and as mentioned above, visualizing, data. Written in Python, but it has interfaces in other languages like MATLAB/Octave, R-Lab \cite{SciPy}.  
    \item \textbf{imutils:} a very good library for image processing, useful for detecting some facial landmark such as the eye, eyebrows and eyelids.
    \item \textbf{Numpy:} fundamental container for mathematical computing in Python \cite{Numpy} and one of the most powerful, useful and fast library for working with multi-dimensional arrays. Also contains a multitude of functions for resizing and manipulate arrays. Most useful when converting an image to an array. 
\end{itemize}
On the setup phase the following steps has to be done: 
\begin{itemize}
    \item download and install cuda with all it's components.
    \item download and install anaconda and than create a virtual environment.
    \item after the steps above were done install all the required framework and libraries.
\end{itemize}

    \subsection{Implementation}
        This section is focused on the practical part of the thesis, mainly on how the focus and drowsiness detector were implemented. In what has to come, the image processing phase and the architecture of the convolutional neural network, followed by how the prediction is made (part of the focus detector), how the drowsiness detector is implemented and lastly, but not the least, how the main program works. Also, the dataset used for the neural network is "AffectNet" which contains around 1 mil images of human faces. \par

\subsubsection{Focus detector} 

\vspace{2ex}
\textbf{Image processing} \par
\vspace{2ex}

This process is useful for detecting the face and the eyes in order to train the neural network model.

\begin{lstlisting}
for image_path in glob('D:/Faculta/licenta/Manually_Annotated_Images/Training/focused/*.*'):
    image1 = cv2.imread(image_path)
    image1 = cv2.cvtColor(image1, cv2.COLOR_BGR2GRAY)
    face1 = face_cascade.detectMultiScale(image1, scaleFactor=1.5, minNeighbors=5)
    roi_eyes1 = ''
    count += 1
    print(1, " ", count, " ", len(face1))

    if len(face1) != 0:
        for (x, y, w, h) in face1:
            roi1 = image1[y:y + h, x:x + w]
            eyes1 = eye_cascade.detectMultiScale(roi1)
            for (ex, ey, ew, eh) in eyes1:
                roi_eyes1 = roi1[ey: ey + eh, ex: ex + ew]
                roi_eyes1 = cv2.resize(roi_eyes1, (60, 60))
        if len(roi_eyes1) != 0:
            image_type_1.append(preprocessing.image.img_to_array(roi_eyes1))
    else:
        eyes1 = eye_cascade.detectMultiScale(image1)
        if len(eyes1) != 0:
            for (ex, ey, ew, eh) in eyes1:
                roi_eyes1 = image1[ey: ey + eh, ex: ex + ew]
                roi_eyes1 = cv2.resize(roi_eyes1, (60, 60))
            image_type_1.append(preprocessing.image.img_to_array(roi_eyes1))
\end{lstlisting}

This algorithm parse through all the photos from the "focused" folder. Than each image is transformed into gray scale image (using OpenCV cvtColor function). After that is applied the face\_cascade which is the OpenCV implementation of the face detection using haar cascades. In the case of detecting the face, than the region of interest is computed (roi = image1[y:y+h,x:x+w]) in order to detect the eyes. After, detect the eyes and compute the region of interest (roi\_eyes1 = roi1[ey: ey + eh, ex: ex + ew]), afterwards the image made of the region of interest is resized to 60 $\times$ 60 which is the expected image size of the network model. Finally, the roi image is stored into an array. If the face is not detected, than the algorithm is trying to detect only the eyes to compute the roi (the same as to the above roi\_eyes1) and resize the image to 60X60 and add it to the array.\par

The same output will be in the case of the "not\_focused" folder, with the specification that the "glob" function contains the path to the "not\_focused" folder. In both cases the result is two arrays which contains the pixels from roi eyes. \par

After all images from those two folders were parsed, the resulted arrays will be transformed into numpy arrays and concatenate them into one numpy array which will contain all the data, the concatenation is done row wise. To make the computations more efficient the elements from the concatenated array will be divided to 255. All of these are done using the below lines of code. 

\begin{lstlisting}
    x_type_focused = np.array(image_type_1)
    x_type_not_focused = np.array(image_type_2)
    
    X = np.concatenate((x_type_focused, x_type_not_focused), axis=0)
    X = X / 255.
\end{lstlisting}

The above code is for setting up the array for the data without any labels. So, before training the model, all of the data have to labeled to make sure that the model will not throw any error, as show in the code below. Where the function to\_categorical will produce the output matrix y, which will contain only the labels for the data. In this case 0 stands for the "focus" label an 1 for "not\_focused" label. Also, the concatenation is done in the same way as for the data, row wise.

\begin{lstlisting}
    y_type_focused = [0 for item in enumerate(x_type_focused)]
    y_type_not_focused = [1 for item in enumerate(x_type_not_focused)]
    y = np.concatenate((y_type_focused, y_type_not_focused), axis=0)
    y = to_categorical(y, num_classes=len(class_name))
\end{lstlisting}

Finally the next step is to train the model using the fit function provided by the Keras framework.

\vspace{2ex}
\textbf{Model architecture} \par
\vspace{2ex}

The architecture of the convolutional neural network is made using Keras. The model is a sequential one, meaning that every layer is executed after the one in front has finished it's job. 

\begin{lstlisting}
    model = Sequential()
    model.add(Conv2D(conv_1, kernel_size=(3, 3), activation='relu', input_shape=(60, 60, color_channels)))
    model.add(MaxPooling2D(pool_size=(2, 2)))
    model.add(Dropout(conv_1_drop))
    
    model.add(Conv2D(conv_2, kernel_size=(3, 3), activation='relu'))
    model.add(MaxPooling2D(pool_size=(2, 2)))
    model.add(Dropout(conv_2_drop))
    
    model.add(Conv2D(128, kernel_size=(3, 3), activation='relu'))
    model.add(MaxPooling2D(pool_size=(2, 2)))
    model.add(Dropout(0.25))
    
    model.add(Flatten())
    
    model.add(Dense(dense_1_n, activation='relu'))
    model.add(Dropout(dense_1_n_drop))
    model.add(Dense(dense_2_n, activation='relu'))
    model.add(Dropout(dense_2_n_drop))
    model.add(Dense(len(class_name), activation='softmax'))
    
    model.compile(optimizer=Adam(lr=lr), loss='binary_crossentropy', metrics=['accuracy'])
\end{lstlisting}

From this model, the first convolution layer is the one that acts as an input layer which accepts images having the shape 60 $\times$ 60 and the color\_channels is equal to 1. The size of the kernel is equal to a tuple with the following values (3,3), which is setting the width and height of the convolution window, remaining the same for the other convolution layers. Also, the activation function is the same across all of the convolution layers (relu, rectified linear unit). It's formula is:

\begin{equation}
    y = max(0,x)
\end{equation}

For the convolution layers, only the filters differs, for the first convolution layer the filter is set to 16, for the second is 32 and for the last one is 128. After each convolution layer there is a maxpooling layer, with the pool\_size= (2,2) meaning that the down sampling is done in two dimensions (vertical, horizontal), followed by a dropout layer which will randomly drop neurons from the convolutional layer. \par

After all these layers, \textbf{flatten} is applied to shrink the dimension to be similar with that of a classical ANN. Afterwards, three fully-connected layers are added to complete the model. First two of them have the 'relu' activation function with 1024 and respectively 512 units. The last one has the 'softmax' activation with 2 units ("focused", "not\_focused"). Also, after every dense layer one dropout layer is added. Finally, the compile function is used, with the following parameters, for optimizer Adam is used and 'binary\_crossentropy' for computing the loss, to make the model ready to be used (see Table \ref{table:modelsummary} for more informations about the model). 

\begin{table}[h!]
    \centering
    \begin{tabular}{||c c c c||} 
    \hline
    Layer & Type & Output Shape & Param \# \\ [0.5ex] 
    \hline\hline
    conv2d\_1 & Conv2D & (None, 58, 58, 16) & 160 \\ [1ex]
    \hline
    max\_pooling2d\_1 & MaxPooling2D & (None, 29, 29, 16) & 0 \\ [1ex]
    \hline 
    dropout\_1 & Dropout & (None,29,29,16) & 0 \\ [1ex]
    \hline
    conv2d\_2 & Conv2D & (None, 29, 29, 32) & 4640 \\ [1ex]
    \hline
    max\_pooling2d\_2 & MaxPooling2D & (None, 14, 14, 32) & 0 \\ [1ex]
    \hline
    dropout\_2 & Dropout & (None, 14, 14, 32) & 0 \\ [1ex]
    \hline
    conv2d\_3 & Conv2D & (None, 14, 14, 128) & 36992 \\ [1ex]
    \hline
    max\_pooling2d\_3 & MaxPooling2D & (None, 7, 7, 128) & 0 \\ [1ex]
    \hline
    dropout\_3 & Dropout & (None, 7, 7, 128) & 0 \\ [1ex]
    \hline
    flatten\_1 & Flatten & (None, 6272) & 0 \\ [1ex]
    \hline
    dense\_1 & Dense & (None, 1024) & 6423552 \\ [1ex]
    \hline
    dropout\_4 & Dropout & (None, 1024) & 0 \\ [1ex]
    \hline
    dense\_2 & Dense & (None, 512) & 524800 \\ [1ex]
    \hline
    dropout\_5 & Dropout & (None, 512) & 0 \\ [1ex]
    \hline
    dense\_3 & Dense & (None, 2) & 1026 \\ [1ex]
    \hline \hline
    Total params: & 6,991,170 & & \\ 
    \hline
    \end{tabular}
    \caption{The summary of the model}
    \label{table:modelsummary}
\end{table}

\vspace{2ex}
\textbf{Prediction} \par
\vspace{2ex}

The last part of the focus detector is making predictions. For this process to work, firstly the model has to be trained and saved. Afterwards, the model is imported and loaded, as show in the code below.

\begin{lstlisting}
   from keras.models import load_model
   
   model = load_model('focus_detector_model1.h5')
\end{lstlisting}

The process of detecting the face and eyes is done in exactly the same way as the algorithm for processing the images, shown at the beginning the Focus Detector implementation. In the case of detecting the face and than the eyes, the region of interest will be computed afterwards it will be resized to be 60X60 and finally the roi will be transformed into arrays of pixels. After the above process, a prediction is made, as shown below.

\begin{lstlisting}
    roi_predict = cv2.resize(roi_gray1, (60, 60))
    roi_predict = preprocessing.image.img_to_array(roi_predict)
    roi_predict = np.expand_dims(roi_predict, axis=0)
    prediction = model.predict(roi_predict)
\end{lstlisting}

In the case that the result of prediction is 'not\_focused' than a counter will be incremented, otherwise it will reset its value. The same scenario happens when the face is not detected. Therefore, if one of the counters exceeds a certain threshold, than an alert will be played. The following lines of code shows the implementation of how the alarm is fired and how he counters get incremented.

\begin{lstlisting}
    if result == 'not_focused':
        COUNTER_PREDICT += 1
    else:
        COUNTER_PREDICT = 0
        
    COUNTER_FOCUS += 1
    if COUNTER_FOCUS >= THRESH_COUNT or COUNTER_PREDICT >= THRESH_PREDICT and SPEED != 0:
        if not ALARM_ON_FOCUS:
            ALARM_ON_FOCUS = True
            t = Thread(target=sound_alarm, args=("alarm.wav",))
            t.daemon = True
            t.start()
        cv2.putText(frame, "FOCUS ALERT!", (10, 30), cv2.FONT_HERSHEY_SIMPLEX, 0.7, (0, 0, 255), 2)
        
    def sound_alarm(path):
        playsound.playsound(path)         
\end{lstlisting}

\subsubsection{Drowsiness Detector}

The \textbf{Drowsiness Detector} is implemented using the following libraries: SciPy, imputils, dlib and OpenCV framework. \par

One of the main parts of this detector is computing the eye aspect ratio which will be used to determine whether the driver will be or not sleepy. For this the code below will compute the proportion of distances between vertical eye landmarks and horizontal eye landmarks.

\begin{lstlisting}
def eye_aspect_ratio(eye):

    a = dist.euclidean(eye[1], eye[5])
    b = dist.euclidean(eye[2], eye[4])

    c = dist.euclidean(eye[0], eye[3])

    ear = (a + b) / (2.0 * c)
    return ear
\end{lstlisting}

In the case of an open or closed eye the value of the eye aspect ratio will remain relatively constant, but if the eye is closed the value will be much smaller thank of that from an open eye. A rapid decrease will be registered in case of blinking. \par

In order to detect the face and eye landmarks, the dlib is used. After the face is detected is applied a "shape\_predictor" which will return the shape of the face. Lastly, to get the eyes from the face landmarks, simply slice the array with the correct indexes, as shown in the below code snippet. 

\begin{lstlisting}
def get_face_utils():
    return dlib.get_frontal_face_detector(), \
           dlib.shape_predictor("shape_predictor_68_face_landmarks.dat"), \
           face_utils.FACIAL_LANDMARKS_IDXS["left_eye"], \
           face_utils.FACIAL_LANDMARKS_IDXS["right_eye"]

\end{lstlisting}

By using the above the returned features,the face is detected, than a for loop is used to parse every face to get the coordinates of the eyes. With them the eye aspect ratio will be computed, one for each eye and than made an average. 

\begin{lstlisting}
rects = detector(gray, 0)

for rect in rects:
    shape = predictor(gray, rect)
    shape = face_utils.shape_to_np(shape)

    leftEye = shape[lStart:lEnd]
    rightEye = shape[rStart:rEnd]
    leftEAR = eye_aspect_ratio(leftEye)
    rightEAR = eye_aspect_ratio(rightEye)

    ear = (leftEAR + rightEAR) / 2.0
\end{lstlisting}

Thus, obtaining the value which will then be compared to a threshold. If it's lower than the threshold for a couple of seconds, then an alarm will be fired to let the driver know he is about to fall asleep. Also, to visualize the result of the computations OpenCv is used. The code snippet below shows the implementation.

\begin{lstlisting}
leftEyeHull = cv2.convexHull(leftEye)
rightEyeHull = cv2.convexHull(rightEye)
cv2.drawContours(frame, [leftEyeHull], -1, (0, 255, 0), 1)
cv2.drawContours(frame, [rightEyeHull], -1, (0, 255, 0), 1)

if ear < EYE_AR_THRESH:
    COUNTER += 1

    if COUNTER >= EYE_AR_CONSEC_FRAMES and SPEED != 0:
        if not ALARM_ON_Drowsiness:
            ALARM_ON_FOCUS = True
            ALARM_ON_Drowsiness = True
            t = Thread(target=sound_alarm, args=("alarm.wav",))
            t.daemon = True
            t.start()
        cv2.putText(frame, "DROWSINESS ALERT!", (10, 30), cv2.FONT_HERSHEY_SIMPLEX, 0.7, (0, 0, 255), 2)
    else:
        COUNTER = 0
        ALARM_ON_Drowsiness = False
         ALARM_ON_FOCUS = False
\end{lstlisting}

\paragraph{Main Program}

Finally, after showing the implementation of the Focus and Drowsiness detector, in the following it will be showcase what the main program contains and how it works. In it, from the focus detector, predictions and how the alert if fired are implemented (their explanation is made in the Implementation section in Focus Detector part). Also, from the Drowsiness detector, only the eye\_aspect\_ratio and get\_face\_utils (functions described in the above section) are not implemented in the main source file. Moreover, the logic which manipulates the webcam and the speed module is also implemented here,along side with all the counters and thresholds. For the speed module, when the user presses the 'w' key the speed will be increased with 10km/h and decreased with the same amount when 's' is pressed, and when the speed is equal to 0 all the counters are set to 0. This is to simulate a car in motion. Lastly, the key 'q' is used to quit the execution of the program and after that destroy all the instances of the webcam. \par

\begin{lstlisting}
# variables for the alarm
ALARM_ON_FOCUS = False
ALARM_ON_Drowsiness = False
ALARM_ON_Predict = False

# variables to count the how many times the driver is not focused/tired
COUNTER_PREDICT = 0
COUNTER_FOCUS = 0
COUNTER = 0

# thresholds that signal danger
THRESH_PREDICT = 20
THRESH_COUNT = 30
EYE_AR_THRESH = 0.3
EYE_AR_CONSEC_FRAMES = 20

SPEED = 0
class_name = ['focused', 'not_focused']
result = ''
model = load_model('focus_detector_model1.h5')

# getting the the face/eye haar cascades and the utils for the face
detector, predictor, (lStart, lEnd), (rStart, rEnd) = get_face_utils()
face_cascade, eye_cascade = get_haar_cascades()
cap = cv2.VideoCapture(0)

while True:
    ret, frame = cap.read()  # get each frame from the webcam
    gray = cv2.cvtColor(frame, cv2.COLOR_BGR2GRAY)  # make the image gray
    faces = face_cascade.detectMultiScale(gray, scaleFactor=1.5, minNeighbors=5)  # detect the face
    .... # prediction and drowsiness implementations 
    cv2.putText(frame, str(SPEED) + " Km/h", (10, 50), cv2.FONT_HERSHEY_SIMPLEX, 0.7, (0, 0, 255), 2)

    if cv2.waitKey(20) & 0xFF == ord('w'):  # simulates when the driver presses on the throttle
        SPEED += 10
    elif cv2.waitKey(20) & 0xFF == ord('s') and SPEED > 0:  # simulates when the driver presses on the brake
        SPEED -= 10

    if SPEED == 0:  # when the speed is 0 reset the counters
        COUNTER_PREDICT = 0
        COUNTER_FOCUS = 0
        COUNTER = 0

    cv2.imshow('frame', frame)
    if cv2.waitKey(20) & 0xFF == ord('q'):
        break

cap.release()
cv2.destroyAllWindows()
\end{lstlisting}

When the program is running, a window will appear in which the face and eyes are detected in real time, the speed is shown and whether one of the thresholds are not met a warning text will appear along side an alarm. 
    \subsection{Testing}
    \subsection{Experiments}

\newpage
\section{Conclusion}
\bibliographystyle{abbrv}
\begin{thebibliography}{1}
    \bibitem{trafficConflict} 
    Vittorio Astarita* and Vincenzo Pasquale Giofré. 
    \textit{From traffic conflict simulation to traffic crash simulation:  introducing traffic safety indicators based on  the explicit simulation of potential driver errors }. 
    Simulation Modelling Practice and Theory. Issue 94, p:215-236, March 2019.
     
    \bibitem{briefAIHistory} 
    Albert Bruce G. Buchanan. 
    \textit{A (Very) Brief History of Artificial Intelligence}.
    Ai Magazine, 26(4):53-60, December 2005.
    
    \bibitem{MLIntro} 
    Alex Smola and S.V.N. Vishwanathan. 
    \textit{Introduction to Machine Learning}.Cambridge University Press, United Kingdom, 2008.
    
    \bibitem{IntelliAgents}
    Russell, Stuart J.; Norvig, Peter, 
    \textit{Artificial Intelligence: A Modern Approach (2nd ed.)}, Upper Saddle River, New Jersey: Prentice Hal, 2003.
    
    \bibitem{ANNBasic}
    Su-Hyun Han, Ko Woon Kim, SangYun Kim, Young Chul Youn.
    \textit{Artificial Neural Network:
    Understanding the Basic Concepts
    without Mathematics}. Dement Neurocognitive Disord. 2018 Sep;17(3):83-89
    
    \bibitem{clinical}
    Ebersole JS, Husain AM, Nordli DR.
    \textit{Current Practice of Clinical Electroencephalography}. 4th ed. Philadelphia,
    PA: Wolters Kluwer, 2014.
    
    \bibitem{NNandML}
    Haykin S, Haykin SS.
    \textit{Neural Networks and Learning Machines}.
     3rd ed. Upper Saddle River, NJ: Prentice Hall, 2009.
     
    \bibitem{ANNbriefOverview}
    Zakaria M, AL-Shebany M, Sarhan S.
    \textit{Artificial neural network: a brief overview}. Int J Eng Res April 2014;4:7-12.
     
    \bibitem{MakeyourOwn}
    Rashid T. \textit{Make Your Own Neural Network}.
    1st ed. North Charleston, SC: CreateSpace Independent Publishing Platform, 2016.
    
    \bibitem{RealTime}
    Rashid T, Huang BQ, Kechadi MT.
    \textit{ A new simple recurrent network with real-time recurrent learning process}. Proceedings of the 14th Irish Conference on Artificial Intelligence \&\ Cognitive Science, AICS
    2003; 2003 September 17–19; Dublin, Ireland. [place unknown]: Artificial Intelligence and Cognitive Science, 2003;169-174.
    
    \bibitem{ClassifNN}
    Kozyrev SV 
    \textit{ Classification by ensembles of neural networks}. p-Adic Numbers Ultrametric Anal Appl
    2012;4:27-33. 
    
    \bibitem{OnlineApprox}
    Bottou L. Chapter 2. On-line learning and stochastic approximations. In: Saad D, editor. 
    \textit{ On-Line Learning in Neural Networks}.  Cambridge: Cambridge University Press, 1999;9-42.
    
    \bibitem{NNTricks}
    Bottou L. Stochastic gradient descent tricks. In: Montavon G, Orr GB, Müller KR, editors.
    \textit{Neural Networks: Tricks of the Trade}.
    2nd ed. Heidelberg: Springer-Verlag Berlin Heidelberg, 2012;421-436.
     
    \bibitem{IntroCNN}
    K.  O’Shea  and  R.  Nash.
    \textit{An introduction to convolutional neural networks}. 2015.
    
    \bibitem{Largescale}
    Karpathy A., Toderici G., Shetty S., Leung T., Sukthankar R., Fei-Fei L.
    \textit{Largescale video classification with convolutional neural networks}. In: Computer Vision and Pattern Recognition. (CVPR), 2014 IEEE Conference on. pp. 1725–1732. IEEE, 2014.
    
    \bibitem{3dconvolutional}
    Ji S., Xu W., Yang M., Yu K.
    \textit{: 3d convolutional neural networks for human action recognition}.
    Pattern Analysis and Machine Intelligence, IEEE Transactions on 35(1), 221–231, 2013.
    
    \bibitem{EvalPooling}
    D. Scherer, A. Muller, and S. Behnke.
    \textit{Evaluation of pooling operations in convolutional architectures for object recognition}. In International Conference on Artificial Neural Networks, 2010.
    
    \bibitem{AutoAI}
    Keshav Bimbraw. 
    \textit{Autonomous cars: Past, present and future a review of the developments in the last century, the present scenario and the expected future of autonomous vehicle technology}. In Proc. 12th Int. Conf. Inform. Control, Automat. Robot.(ICINCO), vol. 1, Jul 2015, pp.191-198.
    
    \bibitem{Vision}
     A. Geiger, P. Lenz, C. Stiller, and R. Urtasun. 
     \textit{Vision meets robotics:
     The kitti dataset}. International Journal of Robotics Research(IJRR), 2013.
     
    \bibitem{Dlib}
    \href{http://dlib.net/}{Dlib Documentation}. \textit{dlib.net}. Retrieved 2019-05-05.
     
     \bibitem{MIT}
     Lex Fridman
     , Daniel E. Brown, Michael Glazer, William Angell, Spencer Dodd, Benedikt Jenik, Jack Terwilliger, Julia Kindelsberger, Li Ding, Sean Seaman, Hillary Abraham, Alea Mehler, Andrew Sipperley, Anthony Pettinato, Bobbie Seppelt, Linda Angell, Bruce Mehler, Bryan Reimer. 
     \textit{MIT Autonomous Vehicle Technology Study: Large-Scale Deep Learning Based Analysis of Driver Behavior and Interaction with Automation}. arXiv preprint arXiv:1711.06976, 30 Sept 2018.
     
    \bibitem{Numpy}
    \href{https://www.numpy.org/}{Numpy Documentation}. \textit{numpy.org}. Retrieved 2019-05-05.
    
    \bibitem{Frust}
    I. Abdic, L. Fridman, D. McDuff, E. Marchi, B. Reimer, and B. Schuller.
    \textit{Driver frustration detection from audio and video in the wild},in KI 2016: Advances in Artificial Intelligence: 39th Annual German Conference on AI, Klagenfurt, Austria, September 26-30, 2016, Proceedings, vol. 9904. Springer, 2016, p. 237.
    
    \bibitem{DetectEmotional}
    H. Gao, A. Yuce, and J.-P. Thiran.
    \textit{“Detecting emotional stress from facial expressions for driving safety}, in Image Processing (ICIP), 2014 IEEE International Conference on. IEEE, 2014, pp. 5961–5965.
    
    \bibitem{EndTOEnd}
    M. Bojarski, D. Del Testa, D. Dworakowski, B. Firner, B. Flepp, P. Goyal, L. D. Jackel, M. Monfort, U. Muller, J. Zhang et al, 
    \textit{End to end learning for self-driving cars}, arXiv preprint arXiv:1604.07316, 2016.
    
    \bibitem{Tensorflow}
    M. Abadi, A. Agarwal et al.,
    \textit{Tensorflow: Large-scale machine learning on heterogeneous distributed systems,} arXiv:1603.04467, 2016.
    
    \bibitem{KerasDoc}
    \href{https://keras.io/#why-this-name-keras}{Keras Documentation}. \textit{keras.io}. Retrieved 2019-05-05.
    
    \bibitem{KerasBack}
    \href{https://keras.io/backend/}{Keras Backend}. \textit{keras.io}. Retrieved 2019-05-05.
    
    \bibitem{KerasWhy}
    \href{https://keras.io/why-use-keras/}{Keras Why}. \textit{keras.io}. Retrieved 2019-05-05.
    
    \bibitem{OpenCV}
    Kari Pulli, Anatoly Baksheev, Kirill Kornyakov, Victor Eruhimov. \textit{Real-time computer vision with OpenCV}. Commun ACM 2019;55(6):61-9.
    
    \bibitem{SciPy}
    \href{https://docs.scipy.org/doc/scipy/reference/tutorial/general.html}{SciPy Documentation}. \textit{scipy.org}. Retrieved 2019-05-05.
    
    \bibitem{Haar}
    Kasinski A, Schmidt A. \textit{The Architecture of the Face and Eyes DetectionSystem Based on Cascade Classifier.} October 2007, DOI: 10.1007/978-3-540-75175-5\textunderscore16, part of the Advances in Soft Computing book series (AINSC, volume 45)
    
    \bibitem{Viola}
    Viola P, Jones M. \textit{Rapid object detection using a boosted cascade of simple features.} 2001, in: Proceedings of CVPR 1:511–518.
    
    \bibitem{Lienhart}
    Lienhart R, Kuranov A, Pisarevsky V. \textit{Empirical Analysis of Detection Cascades of Boosted Classifiers for Rapid Object Detection.} Technical report, Microprocessor Research Lab, Intel Labs, 2002.
    
    \bibitem{Meynet}
    Meynet J, Popovici V, Thiran J. \textit{Face Detection with Mixtures of Boosted Discriminant Features}. Technical report, EPFL, 2005.
    
    \bibitem{Wang}
    Wang Q, Yang J. \textit{Eye Detection in Facial Images with Unconstrained Background}. Journal of Pattern Recognition Research, 1:55–62, 2006.
    
    \bibitem{OpenViola}
    A. Mohsen Abdul Hossen, R. Abd Alsaheb Ogla, M. Mahmood Ali. \textit{Face Detection by Using OpenCV's Viola-Jones Algorithm based on coding eyes}.November 2017,Iraqi Journal of Science 58(58):735-745.
    
    \bibitem{Fernandez}
    Wilson P., Fernandez J. \textit{Facial feature detection using Haar classifiers.} J. Comput. Small Coll. 21:127–133, 2006.
 
    \bibitem{MMIST}
    Yann LeCun, Corinna Cortes, Christopher J.C. Burges. \textit{THE MNIST DATABASE of handwritten digits}.
    \href{http://yann.lecun.com/exdb/mnist/}{MMIST  database}. mnist.
    
    \bibitem{Reinforcement}
    Richard S. Sutton,Andrew G. Barto. \textit{Reinforcement Learning: An Introduction}. The MIT Press Cambridge, Massachusetts London, England. Second edition, in progress, complete draft. January 1, 2018.
    
    \bibitem{Metrics}
    Mohammed Sunasra. \textit{Performance Metrics for Classification problems in Machine Learning}. https://medium.com/thalus-ai/performance-metrics-for-classification-problems-in-machine-learning-part-i-b085d432082b.
    
    \bibitem{Evaluate}
    Aditya Mishra. \textit{Metrics to Evaluate your Machine Learning Algorithm}. https://towardsdatascience.com/metrics-to-evaluate-your-machine-learning-algorithm-f10ba6e38234.
    
    \bibitem{Review}
    Nurulhuda Ismail, Mas Idayu Md. Sabri. \textit{Review  of  Existing  Algorithms  for  Face  Detection  and  Recognition.} Proceedings of the 8th WSEAS International Conference on  Recent Advances in Computational  Intelligence, Man-Machine Systems and Cybernetics, Puerto De La Cruz, Tenerife, Canary Islands,  Spain, December 14-16, 2009, pp. 30-39.
    
    \bibitem{Survey}
    Hiyam Hatem, Zou Beiji, Raed Majeed. \textit{A Survey of Feature Base Methods for Human Face Detection.} International Journal of Control and Automation Vol.8, No.5 (2015), pp.61-78.
\end{thebibliography}

\end{document}
