A series of frameworks and libraries were used in order to develop the application. For the convolutional neural network keras under tensorflow were used, on image processing openCV was the best option and for regular computing and storage a couple of libraries were used, such as dlib, scipy, imutils and numpy. In the following a short description and explanation of why it were used are presented below: 
\begin{itemize}
    \item \textbf{Tensorflow:} the most used library for research and creating neural networks, also contains a lot of math functions. It was created by Google and it's a free and open source software, besides creating neural networks it's good for dataflow and  differentiable programming \cite{Tensorflow}.
    \item \textbf{Keras:} it's a very versatile open source framework, written in Python, capable of creating complex neural networks. It run over a lot of good deep learning frameworks such as Tensorflow, Microsoft Cognitive Toolkitm Theano or PlaidML. The main advantage of Keras is being super user-friendly and fast at the same time. It was developed to improve their research on the ONEIROS project (Open-ended Neuro-Electronic Intelligent Robot Operating System). It's name comes from Greese and it means \textit{horn}  \cite{KerasDoc, KerasBack, KerasWhy}.
    \item \textbf{OpenCV:} or open computer vision is an open source framework, for computing real-time image processing, object recognition and more, developed by Intel \cite{OpenCV}. The main language in which was written is C++, but it has interfaces in other languages like Python, Java and so on.
    \item \textbf{dlib:} a modern and fast library for improving and developing machine learning algorithms. Great for computing the movement of some facial landmarks \cite{Dlib}.
    \item \textbf{SciPy:} it comes as an extension over Numpy containing more mathematical algorithms and classes ideal for data visualization. Very useful for interpreting and as mentioned above, visualizing, data. Written in Python, but it has interfaces in other languages like MATLAB/Octave, R-Lab \cite{SciPy}.  
    \item \textbf{imutils:} a very good library for image processing, useful for detecting some facial landmark such as the eye, eyebrows and eyelids.
    \item \textbf{Numpy:} fundamental container for mathematical computing in Python \cite{Numpy} and one of the most powerful, useful and fast library for working with multi-dimensional arrays. Also contains a multitude of functions for resizing and manipulate arrays. Most useful when converting an image to an array. 
\end{itemize}
On the setup phase the following steps has to be done: 
\begin{itemize}
    \item download and install cuda with all it's components.
    \item download and install anaconda and than create a virtual environment.
    \item after the steps above were done install all the required framework and libraries.
\end{itemize}
