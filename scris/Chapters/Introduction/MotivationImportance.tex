It has been a real problem for the people of the twenty-first century when it comes to being safe while behind the wheel. Since the introduction and popularization of the mobile phone, the majority of the population shifted their attention to the small screen of their phones, thus putting themselves in danger. This habit has quickly made it's way into driving, making the drivers less focused to the road, thus causing more accidents and even worst death of some unfortunate people. A driver has to be focused at many things, such as road signs, pedestrians and many more, by paying more attention to the mobile phone you expose yourself to a possible driver error \cite{trafficConflict}. Also, as the time goes on the automakers keep making cars which are more powerful and faster, thus, taking into account the lack of attention caused by the mobile phones and you have a good recipe for a car accident. As Jeremy Clarkson said “Speed has never killed anyone. Suddenly becoming stationary, that's what gets you.” (goodreads). This may serve as a useful tip for drivers around the world, because not the speed is the bad factor, but the lack of attention to the road and surrounding causes accidents which can lead to dead people.    
\newline
One of the problems to deal with is the rate of distraction and how a specific distraction can turn into a driving error \cite{trafficConflict}. Assuming that a distraction turns out into a specific mistaken driving pattern, than the rate of distraction \ensuremath{\lambda} can be a variable which evaluates the number of distractions that a certain driver can undergo while driving for a fixed period of time \cite{trafficConflict}. The rate \ensuremath{\lambda} is not sufficient to describe the distraction event, so a new variable must be assumed, a distribution \ensuremath{\psi} of the profile of erroneous driving pattern \cite{trafficConflict}. In theory, if the rate \ensuremath{\lambda} and the distribution of \ensuremath{\psi} is known, a Monte Carlo experiment could be adapted to initiate randomly the immediate of distraction and the resulting mistaken driving pattern \cite{trafficConflict}. 
\newline
A fundamental speculation for this rate-of-distraction-based model could be  that the crash frequency is strongly connected to how frequent the driver is distracted, which may be supposed to occur randomly and uniformly across any driven distance \cite{trafficConflict}. In other words, at any moment(or point) of driver's trip, it has the same probability of making an error \cite{trafficConflict}. Nevertheless, some of the crash data have shown that crash rates are higher at specific locations and in specific traffic flow surroundings \cite{trafficConflict}.
\newline
A similar approach has been in the brand new line of cars from BMW, which has a cam in the dashboard that monitorize the behaviour of the driver behind the wheel.
\newpage