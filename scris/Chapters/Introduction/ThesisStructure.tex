The research in this dissertation advances the theory, design and implementation of several particular models. The present work contains 17 bibliographical references and its structured in four chapter as follows:
\begin{itemize}
  \item The first chapter is a short introduction to the problem that this thesis is covering and what objectives are stated. 
  \item The second chapter is a quick overview of how artificial intelligence played an important role in developing the cars of the future.
  \item The third and fourth chapter describes the theoretical part of the thesis which consists in making a short introduction to AI, convolutional neural networks followed by a brief explanation of the influence of the artificial intelligence in automotive and a brief explanation of the technical part of the application. The technical part will parse through how the face and eyes are detected and how the drowsiness detector works.
  \item The fifth chapter covers the implementation of the application which frameworks were used, how the architecture was made and what tests were performed.
  \item The sixth chapter and the last one will be represented by the conclusion of this thesis. \textcolor{green}{\sout{ultimul capitol tb sa fie cel cu conclusions si further work (sectiunea cu bibliografia nu se aminteste in structura tezei; e oarecum implicit ca o lucrare de gen sa aiba biblio :D}}
  
\end{itemize}