The first interaction between an intelligent system and the automotive industry was with the radio controlled car (Houdina Linriccan Wonder). On the rear of a 1926 Chandler were mounted antennae and it could be controlled by another car that sent out radio impulses while coming after the Houdina Wonder. The signals were sent to the "circuit-breakers" which control some small electric motors resulting in directing the car's actions \cite{AutoAI}. This try marked a basic form of what is called today autonomous vehicle. In 1939, General Motors (GM) had sponsored Normal Bel Geddes's Futurama display at the "World's Fair" illustrating an embedded-circuit powered electric car. Like the previous tries, the circuits were planted in roadway and controlled by radio \cite{AutoAI}. See Figure \ref{fig:ex_autoCar}. \par

In year of 1953 on the laboratory floor of the RCA Labs was build a miniature car, that was controlled and guided by cables that were places in a specific pattern. Leland Hancock (state traffic engineer in Nebraska) and L. N. Ress (state engineer) tested the idea of RCA Labs on an actual highway installation. The test took place near the town of Lincoln (Neb), on a 121.92 meters long portion of highway, in the year 1958. On the surface of the road were placed some detector circuits and lights on the edge of the road, that were capable to send impulses to control the car. GM was an important collaborator of the project and equipped two models with sophisticated radio receivers, audible and visual warning devices capable to mimic automatic steering, accelerating and brake control \cite{AutoAI}. \par 

During the 60's the United Kingdom's Transport and Road Research Laboratory tested a driverless car, that went at a speed of 130 km/h, on a test track, without aberrations of speed or direction. Also the car interacted with magnetic wires that were integrated in the road \cite{AutoAI}. \par

\begin{figure}[h!]
    \centering
    \includegraphics[width=1\linewidth]{Images/230px-Linrrican_Wonder.png}
    \caption[Houdina Linriccan Wonder car]{Houdina Linriccan Wonder car \protect\footnotemark}
    \label{fig:ex_autoCar}
\end{figure}

\footnotetext{image taken from: https://en.wikipedia.org/wiki/Houdina\_Radio\_Control\\\#/media/File:Linrrican\_Wonder.png}

After two decades in the 80's, Mercedes-Benz released a robotic van designed by Ernst Dickmanns and his team within the Bundeswehr University Munich, Germany. The robotic van reached a speed of 63 km/h on no traffic roads. Furthermore, different national and international projects, thanks to the progress in autonomous cars, were launched. Between 1987 and 1995, a project named Prometheus conducted by EUREKA was started with one goal in mind, to advance further in the field of autonomous cars. In this project were invested over 1 billion US dollars. Also in the same time period, DARPA (defence advanced research projects agency) of the US Department of Defence had developed the first autonomous land vehicle (AVL) that achieved road-following using computer vision, LIDAR and autonomous control to help the robot to achieve a speed of 31 km/h. These technologies were developed within Carnegie Mellon University, the Environmental Research Institute of Michigan, University of Maryland, Martin Marietta and SRI International. \cite{AutoAI}. \par

One of the most important projects of the late twenty century happened in 1995, the Carnegie Mellon University’s Navlab project had achieved an autonomous driving percentage of 98.2\%\ on a 5,000 km trip, known as "No Hands Across America" or NHOA. Although the car was not fully-automated, only the control of the steering wheel was done by using neural networks, but the throttle and brake were operated by a human \cite{AutoAI}. \par  

One year later, Alberto Broggi started the ARGO Project. The goal of the project was to travel 1900 km in six days in northen Italy, just following the the painted lane marks and with an average speed of 90 km/h. Overall, the car managed to be 94\%\ of it's journey in fully-automatic mode and the longest automatic stretch of 55 km \cite{AutoAI}. \par 

In the early 2000s the US government start implementing and funding autonomous military vehicles, such as Demo I, Demo II and Demo III. These autonomous robots are able to travel kilometers on difficult terrain, detect and avoid obstacles like rocks, trees etc, making the soldiers life easy \cite{AutoAI}. \par

The year of 2010 mark the unveiling of the first 2 seat electric car that could be operated manually or autonomously, at the Expo 2010 in Shanghai. GMs 2 seat electric car, know as Electric Networked Vehicle, was divided into three distinct vehicle types. GM's ENV quickly became an important asset in making vehicles more connected with each other and motion control algorithms. From July 20, 2010 to October 2, 2010, the VIAC or VisLab Intercontinental Autonomous Challenge took place. This challenge was a 13,000 km trip from Parma, Italy to Shanghai, China. The goal of this prject was to show that in the future goods could be transported between continents with vehicle that require slight intervention by the human. Another, important, milestone was the Audi's TTS autonomous research car, that completed, in September 2010, Pike's Peak in around 27 minutes, fully autonomous, coming about 10 minutes of delay to the human record of 17 minutes. This was done by using a similar approach to the auto pilot feature of the airplanes. Also, the project was sponsored by the EU as part of the HAVEit (Highly Automated Vehicles for Intelligent Transport) project \cite{AutoAI}. \par

The first car that was licensed on the public streets and highways of the German state of Berlin were "MadeInGermany" and "Spirit of Berlin", both developed by the AutoNOMOs Labs. Some of the technologies developed for this project were: driver assistance system, an innovative safety system. It was equipped with an accurate GPS unit and three laser in the front and back that could detect pedestrians or cars 360 degrees, traffic lights, intercity traffic and roundabouts. Those from Daimler R  \&\ D in collaboration with Karlsruhe Institute of Technology made a Mercedes S-Klasse capable to drive autonomously for a distance of 100 km from MannHein to Pforzheim, Germany. It used a lot of technology such as next generation radars and stereo cameras which assisted in its "autonomous automation" and cutting edge machine vision algorithms. Also, Toyota had developed an autonomous car focused in negating the crashes, they used ITS, Intelligent Transport Systems, technology. The systems of the car were engineered in such a way that in case of a failure the car will not crash. Although the car is not fully-autonomous, the human being able to intervenes at any time \cite{AutoAI}.  \par

BRAiVE, one of the autonomous cars that participated in VIsLab, drove, in July 12th 2013, in the center of Parma. It was the first fully-autonomous car that navigated on rural roads, crosswalks, traffic lights, pedestrian areas, roundabouts and artificial hazards \cite{AutoAI}. \par

The Induct Technology from France had made a robotically guided electric shuttle named "Navia". The robot can reach a maximum speed of 20 km/h and can fit 10 passengers. It was tested in some universities across England, Singapore and Switzerland. It was capable to generate a 3D map of the surrounds thanks to stereoscopic cameras and four LIDAR units \cite{AutoAI}. Lastly the giant Google had make his way into developing and improving the software, in order to create fully-autonomous cars capable to surpass the humans and make the roads more safer than ever. \par

The focus nowadays, in the driving domain, is the design of perception and control systems, that uses learning based approaches with large data collections and annotations in order to build models that are able to operate over edge cases of some real-world operations. Thus, making the deep learning research in focuses to address detection, estimation, prediction, labeling, generation, control and tasks planning. Some of the current technologies in developing are "body pose estimation", "semantic scene perception", "fine-grained face recognition" and "driving state prediction" \cite{Vision, MIT}. These tasks are of a great importance because it can lead to understand how to mimic the human posture and to create an awareness based on the current environment. In what is to come, it will be briefly summarized the above states technologies as follows: 

\begin{itemize}
    \item \textbf{Fine-grained Face Recognition}: This type of face recognition take a step forward, ahead of classical face recognition techniques, by focusing on understanding the behavior of a human toward recognizing the face, like facial expression or detect the eye gaze. This technique is very useful to explore, in depth, the power of predicting the driver eyeballs in a way that makes the driving more secure and the driving experience more pleasing \cite{MIT, Frust, DetectEmotional}.  
    
    \item \textbf{Body Pose Estimation}: By using this approach it can enhance the performance, efficiency and understanding of a lot of real-world usage for robots and action recognition. Some of the fortunate approaches range from the usage of "depth image",via DNN (deep neural network) or the usage CNN (convolutional neural network) and graphical models. This technique is especially useful for driving, which detects the driver posture, presented by the "skeleton data", that includes the position of the wrist, shoulder joints and elbow in order to model the behavior of the humans when driving. Thus, monitoring the vigilance of the driver through "visual analysis of eye state and head pose" \cite{MIT}.
    
    \item \textbf{Semantic Scene Perception}: Interpreting and understanding the picture from a 2D image has been a long and challenging task in computer vision, this, usually is referring to the autopsy of the semantic image. By using datasets of  large proportions (datesets of places, cityscapes etc) and mighty deep learning approaches obtaining some state-of-the-art results. Thus, the study of precise "driving scene perception" is nowadays actively considered in both universities and automotive industry \cite{MIT, EndTOEnd}.
    
    \item \textbf{Driving State Prediction}: Represents the goal of autonomous driving, the state of the vehicle is commonly considered as a "direct illustration of human decision" while driving. For this, the machine learning approach serves as the "ground truth" for a lot of distinct viewpoints, like driving behavior, steering commands, controlling the speed etc \cite{MIT, EndTOEnd}.
\end{itemize}

In the modern era of vehicles thinks like driver assistance, vehicle performance and driver experience are more and more automated using learning-based ways, thanks to more datasets being released that could be used by communities of researchers. One study from MIT, MIT-AVT, that aims to be the base of countless such datasets, which helps developing and train new neural network architectures in order to bring current and future solutions for, a lot of standard and 
mixed subtasks, that helps semi-autonomous or fully-autonomous driving be more safe and affordable \cite{MIT}.