The roots of Artificial intelligence, and the concept of smart machines, can be found in the Greek mythology. Some intelligent devices have appeared, since then, in the literature, that have demonstrated to behave with a degree of intelligence. Following the World War II which marks the beginning of the so called era of "modern computers". This machines were able to create programs that perform difficult intellectual tasks. Using these programs, humans have created general tools with a wide variety of applications in everyday problems \cite{briefAIHistory}, see Figure \ref{fig:SRI_Robot} and \ref{fig:Unimate_Robot}.

\begin{figure}[h!]
    \centering
    \includegraphics[width=.3\linewidth]{Images/SRI_Shakey.jpg}
    \caption[Shakey the Robot]{Shakey the Robot was the first general-purpose mobile robot to be able to reason about its own actions. While other robots would have to be instructed on each individual step of completing a larger task, Shakey could analyze commands and break them down into basic chunks by itself.\protect\footnotemark}
    \label{fig:SRI_Robot}
\end{figure}

\footnotetext{image taken from: https://en.wikipedia.org/wiki/Shakey\_the\_robot\#/media/File:\\SRI\_Shakey\_with\_callouts.jpg}

\begin{figure}[h!]
    \centering
    \includegraphics[width=.4\linewidth]{Images/UnimateRobot.jpg}
    \caption[Unimate]{The Unimate was the first industrial robot, used at General Motors assembly line. It was invented by George Devol in the 1950s.\protect\footnotemark}
    \label{fig:Unimate_Robot}
\end{figure}

\footnotetext{image taken from: https://www.robotics.org/joseph-engelberger/unimate.cfm}

As stated above, artificial intelligence is the intelligence that is manifested by machines, in contrast to the natural intelligence displayed by humans or animals. In computer science the research of artificial intelligence is defined as the study of 'intelligent agents' \cite{IntelliAgents}. Moreover artificial intelligence can be classified into three categories of systems.: analytical, human-inspired and humanized AI. The first one, Analytical AI, generates a cognitive image of the world by learning based on past experiences to predict future choices.  Human-inspired AI is able to understand human emotions, in addition to the cognitive ones, by combining elements from both cognitive and emotional intelligence. The last type is called Humanized AI which is a machine capable of being self-conscious and self-aware in interactions with others, having a all types of competencies (i.e. cognitive, emotional and social intelligence) like his human counterpart. \par

One of the main core part of the artificial intelligence is Machine learning. Most of the science of a machine learning system is to solve problems and give good insurance for the end result \cite{MLIntro}. Learning without any kind of supervision requires an ability to identify patterns in streams of input. \par

Some of the most common use of the machine learning is in the implementation of a search engine e.g the Google search from Google. This means that when searching on Google, the engine behind receives a query and returns the relevant webpages. To achieve this goal, any search engine has to 'know' which pages to return based on their relevancy and query (see Figure \ref{fig:SearchEngine}). This knowledge can be gained from two main sources: the first one is based on the link structure of webpage, content and the frequency of what the users will follow on the suggested links in a query, and the second source is to compare the entered query with existing queries. For a more efficient search, machine learning is used increasingly to automate the process of designing a search engine \cite{MLIntro}. \par

\textcolor{green}{\sout{ai te rog grija cu parantezele: inainte de paranteza se lasa un spatiu liber, dupa paranteza nu se lasa; cateva situatii am corectat eu, dar te rog verifica tot textul}}

Other applications that take advantage of machine learning are speech recognition (being able to translate the speech in text, such as the Google service when shout 'Ok Google'), finger print recognition (for security purpose, being able to know which finger print is, widely use in securing phones), safety features on modern cars (such as adaptive cruise control, being able to adapt the speed based on the car in front,  lane keep, the ability to stay within the same lane and adaptive headlights, the capability of the lights to turn of when detecting a incoming car to not blind the other driver), making the behaviour of the avatar in computer games (i.e Formula 1) \cite{MLIntro}. \par

\textcolor{green}{\sout{ucred ca ar merita sa amintesti pe scurt si de cele 3 directii mari din ML: supervised learning, unsupervised learning si reinforcement learning}} \par
In machine learning there are three main paradigms:

\begin{itemize}
    \item The first paradigm is \textbf{supervised learning}, which translated in real world it assumes only knowing the starting and ending point of a journey, and the route will be determined progressing. So in supervised learning the training is done through pre-labelled inputs. Each training sample has a set of input values and multiple (one or more) associated labeled output values. The goal of this technique is to cut down the classification error, by correct computation of the resulting value of a specific training example, in the training phase. \cite{IntroCNN}.

    \item The second is \textbf{unsupervised learning} paradigm, which means that the input data is no more labeled. Therefore, the success is given whether the network can reduce or increase the linked cost function. However, for a network that uses an unsupervised learning technique it's important to note that in order for this paradigm to work properly, a supervised learning must be done for small portion of the given data \cite{IntroCNN}.  
    \item The last important paradigm is \textbf{reinforcement learning} which represents the learning of what to do, how to tie up situations to actions, in order to boost the numerical knowledge. Thus, unlike supervised or unsupervised learning, the output is given by trying the actions that gives the better rewards. In reinforcement learning, the trial and error and delayed reward are the most essential features \cite{Reinforcement}. 
\end{itemize}

The main difference between a machine learning and a computer program, consists of what the end result will look like. In other words, a computer program takes data as input, processes them and finally gives an output or result.Also a programmer will impose how the data is processed. On the other hand in machine learning, the input and output data is given, and the machine will find the processes with which the given input achieves the given output data. Moreover, using this process the machine can predict the unknown output, when new input data is given \cite{ANNBasic}.

The number of problems that can be solved using learning is by no means small, as stated in the lines above. In other words, the number of templates identified by the researchers which makes the deployment of machine learning quite easy is growing rapidly. In the following lines there is a list by no means complete of templates \cite{MLIntro}:

\textcolor{green}{Enumerarile de mai jos ar putea fi formulate folosind aceleasi filtre:\\
1. ce se da si ce se cere\\
2. care sunt diferentele si asemanarile intre ele\\
\\
De exemplu au as vorbi mai intai de specificarea unei probleme de ML: se dau date (in si out) si se cere identificarea unei legaturi/pattern intre in si out; apoi as zice ca la clasificare out-ul sunt label-uri (2 --- binary classification --- sau mai multe --- multiclass classification), iar la regresie out-ul este numeric; apoi la structured estimation out-ul e reprezentat de laebl-uri si inca ceva informatii suplimentare
}

\begin{itemize}
  \item \textbf{Binary Classification} probably the most often studied problem in machine learning, which has led to a large number of relevant algorithmic and theoretic improvements over the last century \cite{MLIntro}.
  \item \textbf{Multiclass Classification} which represents an extension of the binary classification. The main difference is that in the case of multiclass there are a range of different values \cite{MLIntro}.
  \item \textbf{Structured Estimation} goes a step above the simple multiclass estimation by taking into account that labels have secondary structure which can help in the estimation process \cite{MLIntro}.
  \item \textbf{Regression} presents another prototypical application. The goal of this template is to estimate a real-value variable for which a pattern is given \cite{MLIntro}.
  \item \textbf{Novelty Detection} is a bit vague template. Its used to describe the problem of resolve 'unusual' perception given a set of past measurements \cite{MLIntro}.
\end{itemize}

\begin{figure}[h!]
    \centering
    \includegraphics[width=.8\linewidth]{Images/SearchEngine.png}
    \caption{The 5 top scoring sites from the query 'machine learning' \cite{MLIntro}}
    \label{fig:SearchEngine}
\end{figure}

In order to measure and evaluate the performance of a machine learning algorithm, some performance metrics can be used. Every metrics will influence, in a different way, how the machine learning algorithm will be measured and compared. 

The first metric is called \textbf{Confusion Matrix}, being the easiest one. It is used for discovering the accuracy and faultlessness of a model. Moreover, is good for classification problems in which the output is of two or more types of classes \cite{Metrics}. Terms that correlate with the first metric (for simplicity 1 is True and 0 is False for a binary classification): 

\begin{figure}[h!]
    \centering
    \includegraphics[width=.6\linewidth]{Images/ConfusionMatrix.png}
    \caption{Confusion Matrix \cite{Metrics}}
    \label{fig:ConfusionMatrix}
\end{figure}

\begin{enumerate}
    \item \textbf{True Positive (TP):} occurs when the expected output is \textbf{1} and the value predicted is also \textbf{1} \cite{Metrics}.
    \item \textbf{True Negative (TN):} manifests when the expected output is \textbf{0} and the result of the prediction is \textbf{0} \cite{Metrics}.
    \item \textbf{False Positive (FP):} represents a mismatch between the actual result of the class and the result of the prediction, namely the expected output is \textbf{0} and the prediction gives \textbf{1} \cite{Metrics}.
    \item \textbf{False Negative (FN):} is also a mismatch of the actual result and the predicted one, namely the expected value is \textbf{1} and the predicted is \textbf{0} \cite{Metrics}.
\end{enumerate}

The second metric, which is very useful, is \textbf{Accuracy}. In classification problems represents the number of correctly made predictions over the total number of predictions \cite{Metrics}.

\begin{figure}[h!]
    \centering
    \includegraphics[width=.6\linewidth]{Images/Accuracy.png}
    \caption{Accuracy \cite{Metrics}}
    \label{fig:Accuracy}
\end{figure}

Where at the nominator are the correct prediction, whereas at the denominator are all the predictions made \cite{Metrics}.

\textbf{Precision} is the next performance metric that measures the proportion of the data that has been predicted as True (TP and FP) and the data that are actually true (TP) \cite{Metrics}. 

\begin{figure}[h!]
    \centering
    \includegraphics[width=.6\linewidth]{Images/Precision.png}
    \caption{Precision \cite{Metrics}}
    \label{fig:Precision}
\end{figure}

\textbf{Recall or Sensitivity} is another metric that measures the proportion of data that is actually True was determine by the algorithm as True. The positives (TP and FN) and the data 'diagnosed' by the model as True (TP) \cite{Metrics}.

\begin{figure}[h!]
    \centering
    \includegraphics[width=.6\linewidth]{Images/Recall.png}
    \caption{Recall or Sensitivity \cite{Metrics}}
    \label{fig:Recall}
\end{figure}

\textbf{F1 Score} represents the Harmonic Mean betwixt precision and recall. It tells how clear-cut the classifier is and how powerful it is \cite{Evaluate}. Mathematically, it can be declare as follows:

\begin{center}
    \begin{equation}
        F1 =2 * \frac{1}{\frac{1}{precision} + \frac{1}{recall}}
    \end{equation}
\end{center}

\textbf{Mean Absolute Error (MAE)} represents the mean of the difference betwixt the original and predicted values. It returns the measurement of how distant are the predictions from the expected output. \textbf{Mean Square Error (MSE)} is very similar to Mean Absolute Error with the difference that the MSE takes the mean of the square of the computed difference betwixt the original and predicted values. For computing the gradient MSE is more suited than MAE, because does not require complex computations \cite{Evaluate}.

\textcolor{green}{\sout{eu as aminti acest aspect legat de date si process inainte de clasificare problemelor de ML si as include cu o enumerare a algoritmilor care pot fi folositi in ML, precizand ca in sectiunea urmatoare se va detalia alg de ANN, iar apoi as incheia sectiunea cu o prezentare a metricilor de performanta care se folosesc in ML (MSError, Accuracy Precision, Recall, F-measure, True positive, False positive, etc. - metrici independente de alg de learning folosit in rezolvarea unei probleme}}
