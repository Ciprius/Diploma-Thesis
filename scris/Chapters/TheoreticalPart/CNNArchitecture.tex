As stated in the lines above a convolutional neural network focuses primarily on receiving as input images. Thus, the architecture has to be set up in a way to reflect the necessity of dealing with this type of data (dataset of images). Therefore, one of the main difference between the neurons from an ANN and the neurons from a CNN is that the neurons are organised into three dimensions (height, width and depth representing the spatial dimension). Taking as an example, the input will have the following shape 64 $\times$ 64 $\times$ 3 (height,width and depth), meaning that the final result would have the dimension 1 $\times$ 1 $\times$ \textit{n} (n being the number of possible classes). Whereas in a traditional ANN the input would contain only one dimension, the weight \cite{IntroCNN, Largescale}. \par

In what is has to come, it will shown what are the layers from a CNN and what are purpose. Therefore, a convolutional neural network is composed of three types of layers: convolutional layer, pooling layer and fully-connected layer \cite{IntroCNN}. A simplified architecture for classifying images from MMIST dataset \cite{MMIST} is shown in the Figure \ref{fig:ex_mmistarchitecture}.

\begin{figure}[h!]
    \centering
    \includegraphics[width=1.1\linewidth]{Images/MMISTarchitecture.png}
    \caption{A simple convolutional NN with five layers \cite{IntroCNN}}
    \label{fig:ex_mmistarchitecture}
\end{figure}

Taken the example of the CNN shown in the Figure \ref{fig:ex_mmistarchitecture}, the basic functionality can be separated into four key zones. \par

\begin{enumerate}
    \item the first layer is the \textbf{input layer} which holds the pixel values of an image \cite{IntroCNN}.
    \item the second layer is the \textbf{convolution layer} which finds the output value of the neurons which were connected to the local regions of the input by computing the scalar product betwixt their weights and the zone connected to the input volume \cite{IntroCNN}.
    \item The purpose of the \textbf{pooling layer} is to perform a downsampling of the spatial dimension of the input, thus reducing the number of parameters \cite{IntroCNN}.
    \item Last but not least, there are the \textbf{fully-connected layers}. Their duty is the same as in standard ANNs and pursuit to produce, from the activations, classes of scores which are used for classification \cite{IntroCNN}.
\end{enumerate}

By using this transformation method, CNNs can transform from input layer, using convolutional and downsampling, to class scores that are used in classification and regression \cite{IntroCNN}. The following three sections will show a deeper explanation of the main layers of a convolutional network.